\documentclass[12pt, a4paper]{article}

\usepackage[english]{babel}
\usepackage[utf8]{inputenc}
\usepackage[T1]{fontenc}
\usepackage{amsmath, amssymb, amsthm, mathrsfs}
\usepackage{physics}
\usepackage{microtype}
\usepackage{siunitx}
\sisetup{
    output-decimal-marker = {.},
    group-separator = {,},
    per-mode = symbol
}

\usepackage{graphicx, wrapfig, booktabs, multirow, multicol, xcolor}
\usepackage{geometry, abstract, caption, subcaption}
\usepackage{listings}
\usepackage[hyphens]{url}
\usepackage{hyperref}
\usepackage{appendix}
\hypersetup{
    colorlinks=true,
    linkcolor=blue,
    filecolor=magenta,
    urlcolor=cyan,
    pdftitle={Quantum Simulation of the Bardo Thodol},
    pdfauthor={Horacio Hector Hamann},
    pdfsubject={Quantum Computation and Buddhist Philosophy},
    pdfkeywords={Bardo Thodol, Quantum Computing, Qutrits, Consciousness, Simulation},
    breaklinks=true,
    pdfencoding=auto
}
\usepackage{fancyhdr}

% Quantikz after physics
\usepackage{tikz}
\usetikzlibrary{quantikz}

% Geometry configuration
\geometry{left=2.5cm, right=2.5cm, top=2.5cm, bottom=2.5cm}

% Page numbering solution
\setlength{\headheight}{14.5pt}

% Custom colors
\definecolor{deepblue}{rgb}{0,0.2,0.6}
\definecolor{deepred}{rgb}{0.6,0,0}
\definecolor{deepgreen}{rgb}{0,0.5,0}
\definecolor{codegray}{rgb}{0.5,0.5,0.5}
\definecolor{paradoxcolor}{rgb}{0.8,0.4,0}

% Listings configuration
\lstset{
    language=Python,
    basicstyle=\ttfamily\footnotesize,
    keywordstyle=\color{deepblue}\bfseries,
    commentstyle=\color{deepgreen}\itshape,
    stringstyle=\color{deepred},
    numbers=left,
    numberstyle=\tiny\color{codegray},
    stepnumber=1,
    numbersep=8pt,
    backgroundcolor=\color{white},
    showspaces=false,
    showstringspaces=false,
    showtabs=false,
    tabsize=4,
    captionpos=b,
    breaklines=true,
    breakatwhitespace=true,
    breakindent=20pt,
    frame=single,
    rulecolor=\color{codegray},
    xleftmargin=17pt,
    framexleftmargin=17pt,
    escapeinside={(*@}{@*)},
    inputencoding=utf8,
    extendedchars=true
}

% Header configuration
\pagestyle{fancy}
\fancyhf{}
\fancyhead[L]{\small Quantum Simulation of the Bardo Thodol}
\fancyhead[R]{\small \thepage}
\renewcommand{\headrulewidth}{0.4pt}

% COMMANDS
\newcommand{\expected}[1]{\langle #1 \rangle}
\newcommand{\state}[1]{\ket{\psi_{#1}}}
\newcommand{\hamiltonian}{\hat{H}}
\newcommand{\unitary}{\hat{U}}
\newcommand{\karmaop}{\hat{K}}
\newcommand{\vacuumstate}{\ket{2}}
\newcommand{\manifeststate}{\ket{0}}
\newcommand{\potentialstate}{\ket{1}}

% Theorems and definitions
\newtheorem{theorem}{Theorem}
\newtheorem{definition}{Definition}
\newtheorem{corollary}{Corollary}
\newtheorem{lemma}{Lemma}
\newtheorem{paradox}{Paradox}[section]

% Epistemological warning environment
\newenvironment{epistemic_warning}[1]{
    \begin{center}
    \begin{minipage}{0.9\textwidth}
    \color{paradoxcolor}
    \hrule
    \vspace{0.2cm}
    \textbf{Level} & \textbf{What it affirms} & \textbf{What it does NOT affirm} \\
\midrule
\textbf{Conventional} (\emph{saṃvṛti}) & Quantum mathematics formally valid & That they describe ultimate reality \\
\textbf{Ultimate} (\emph{paramārtha}) & Phenomena lack inherent nature & That mathematics are useless \\
\textbf{Pedagogical} (\emph{upāya}) & Model useful for exploring structures & That it is ontological description \\
\bottomrule
\end{tabular}
\end{table}

\subsection{Future Directions}

\begin{itemize}
    \item \textbf{Experimental Validation}: Integration with advanced meditation and EEG data, \textit{with warning that neural correlates are not experience}
    \item \textbf{Quantum Hardware}: Implementation on real quantum processors (IBM Q, Rigetti), \textit{as computational feasibility demonstration}
    \item \textbf{Extended Models}: Generalization to higher-dimensional systems, \textit{maintaining epistemological transparency}
    \item \textbf{Clinical Applications}: Potential applications in therapy and altered consciousness states, \textit{without psychologistic reductionism}
    \item \textbf{Contemplative-scientific dialogue}: Validation with advanced practitioners on model's heuristic utility
\end{itemize}

\subsection{Scientific and Philosophical Impact}

This work establishes a bridge between ancestral contemplative wisdom and modern computational science, opening new avenues for interdisciplinary research in:

\begin{itemize}
    \item Philosophy of mind and cognitive science \textit{with non-reductionist epistemology}
    \item Quantum computing and information theory \textit{applied to phenomenology}
    \item Contemplative studies and neurophenomenology \textit{with respect to experiential irreducibility}
    \item \textbf{Reflexive meta-modeling}: Models incorporating self-criticism
\end{itemize}

\subsection{Final Reflection: The Finger and the Moon}

As taught in the \emph{Laṅkāvatāra Sūtra}:

\begin{quote}
\textit{"Words and teachings are like a finger pointing at the moon. The finger can indicate where the moon is, but the finger is not the moon. To see the moon, one must look beyond the finger."}
\end{quote}

This computational model is the finger. Direct experience of Bardo states is the moon. Not confusing one with the other is the wisdom enabling effective model use.

% Appendices
\begin{appendices}
\section{Complete Code Implementation}

\subsection{Main System Class}

\begin{lstlisting}[caption={Complete BardoQuantumSystem}, label={lst:complete_system}]
class QuantumMetrics:
    """Class for calculating advanced quantum metrics"""

    @staticmethod
    def coherence(state):
        """Calculates quantum coherence (l1 norm off-diagonal)"""
        if state.type == 'ket':
            rho = state * state.dag()
        else:
            rho = state
        rho_array = rho.full()
        n = rho_array.shape[0]
        coh = 0.0
        for i in range(n):
            for j in range(n):
                if i != j:
                    coh += abs(rho_array[i, j])
        return coh

    @staticmethod
    def purity(state):
        """Calculates state purity: Tr(rho^2)"""
        if state.type == 'ket':
            return 1.0
        else:
            rho = state
            return (rho * rho).tr().real

    @staticmethod
    def von_neumann_entropy(state):
        """Calculates Von Neumann entropy: -Tr(rho log2 rho)"""
        if state.type == 'ket':
            rho = state * state.dag()
        else:
            rho = state
        eigvals = rho.eigenvalues()
        entropy = 0.0
        for v in eigvals:
            if v > 0:
                entropy -= v * np.log2(v)
        return entropy


class QuantumAnalytics:
    """Centralized quantum analysis system"""

    @staticmethod
    def analyze_transitions(probabilities, threshold=0.1):
        """Unified analysis of state transitions"""
        probs = np.array(probabilities)
        transitions = []

        for i in range(1, len(probs)):
            changes = np.abs(probs[i] - probs[i-1])
            max_change = np.max(changes)

            if max_change > threshold:
                transitions.append({
                    'time_index': i,
                    'magnitude': float(max_change),
                    'from_state': int(np.argmax(probs[i-1])),
                    'to_state': int(np.argmax(probs[i])),
                    'change_vector': changes.tolist()
                })

        return transitions

    @staticmethod
    def find_dominant_state(probabilities):
        """Unified dominant state analysis"""
        probs = np.array(probabilities)
        dominant_states = np.argmax(probs, axis=1)
        total_steps = len(dominant_states)

        return {
            'dominant_states': dominant_states.tolist(),
            'time_in_samsara': int(np.sum(dominant_states == 0)),
            'time_in_karmic': int(np.sum(dominant_states == 1)),
            'time_in_void': int(np.sum(dominant_states == 2)),
            'dominance_ratio': {
                'samsara': float(np.sum(dominant_states == 0) / total_steps),
                'karmic': float(np.sum(dominant_states == 1) / total_steps),
                'void': float(np.sum(dominant_states == 2) / total_steps)
            }
        }


class BardoQuantumSystem:
    """
    Complete quantum simulation system for Bardo Thodol
    WITH EXPLICIT LIMITATION DOCUMENTATION
    """

    def __init__(self, **parameters):
        self.set_parameters(parameters)
        self.initialize_quantum_system()
        self.metrics = QuantumMetrics()
        self.analytics = QuantumAnalytics()

        # Document model paradoxes
        self.epistemic_warnings = {
            'karma_quantification':
                'Numerical parameters reify karma (Paradox 1)',
            'sunyata_vector':
                'Vector |2> reifies emptiness (Paradox 2)',
            'temporal_parameter':
                'Time t is mathematical convention (Paradox 3)',
            'measurement_duality':
                'Maintains subject-object framework (Paradox 4)'
        }

    def set_parameters(self, params):
        """Configure system parameters"""
        self.karma_params = params.get('karma_params', {
            'clarity': 0.8,
            'attachment': 0.3,
            'compassion': 0.9,
            'wisdom': 0.7
        })
        self.time_parameters = params.get('time_params', {
            'total_time': 4*np.pi,
            'steps': 1000
        })

    def initialize_quantum_system(self):
        """Initialize base quantum system"""
        self.dimension = 3
        self.states = {
            'samsara': qt.basis(3, 0),
            'karmic': qt.basis(3, 1),
            'void': qt.basis(3, 2)
        }
        self.operators = self._create_operators()
        self.current_state = self.states['void']

    def _create_operators(self):
        """Create quantum operators for system"""
        # Projection operators
        P0 = qt.basis(3, 0) * qt.basis(3, 0).dag()
        P1 = qt.basis(3, 1) * qt.basis(3, 1).dag()
        P2 = qt.basis(3, 2) * qt.basis(3, 2).dag()

        # Transition operators
        S01 = qt.basis(3, 0) * qt.basis(3, 1).dag()
        S12 = qt.basis(3, 1) * qt.basis(3, 2).dag()
        S20 = qt.basis(3, 2) * qt.basis(3, 0).dag()

        # Base Hamiltonian
        H0 = P0 * 0.1 + P1 * 0.2 + P2 * 0.3

        # Karmic operator
        K = self.karma_params['attachment'] * (S01 + S01.dag()) + \
            self.karma_params['clarity'] * (S12 + S12.dag()) + \
            self.karma_params['compassion'] * (S20 + S20.dag())

        return {
            'P0': P0, 'P1': P1, 'P2': P2,
            'S01': S01, 'S12': S12, 'S20': S20,
            'H0': H0, 'K': K
        }

    def simulate_bardo_transition(self, time_steps=1000,
                                  attention_function='logistic'):
        """Simulate complete transition"""
        times = np.linspace(0, self.time_parameters['total_time'],
                          time_steps)
        results = {
            'probabilities': [],
            'coherence': [],
            'purity': [],
            'states': []
        }

        current_state = self.current_state

        for t in times:
            attention = self._attention_evolution(t, attention_function)
            H_eff = self.operators['H0'] + attention * self.operators['K']
            U = (-1j * t * H_eff).expm()
            evolved_state = U * current_state

            probs = [qt.expect(self.operators[f'P{i}'], evolved_state)
                    for i in range(3)]
            coherence = self.metrics.coherence(evolved_state)
            purity = self.metrics.purity(evolved_state)

            results['probabilities'].append(probs)
            results['coherence'].append(coherence)
            results['purity'].append(purity)
            results['states'].append(evolved_state)

            current_state = evolved_state

        return results, times

    def _attention_evolution(self, t, attention_function='logistic'):
        """Attention evolution over time"""
        if attention_function == 'logistic':
            return 1.0 / (1.0 + np.exp(-0.5 * (t - 2*np.pi)))
        elif attention_function == 'sinusoidal':
            return 0.5 * (1.0 + np.sin(t))
        else:
            return 1.0

    def run_complete_simulation(self):
        """Execute complete simulation with analysis"""
        results, times = self.simulate_bardo_transition()
        probs_array = np.array(results['probabilities'])

        analysis_report = {
            'final_state_classification': self._classify_final_state(
                results['states'][-1]
            ),
            'transitions': self.analytics.analyze_transitions(probs_array),
            'dominant_state_analysis':
                self.analytics.find_dominant_state(probs_array),
            'quantum_metrics': {
                'avg_coherence': float(np.mean(results['coherence'])),
                'avg_purity': float(np.mean(results['purity'])),
                'final_entropy': self.metrics.von_neumann_entropy(
                    results['states'][-1]
                )
            },
            'epistemic_warnings': self.epistemic_warnings
        }

        return results, times, analysis_report

    def _classify_final_state(self, state):
        """Classify final state by probabilities"""
        probs = [float(qt.expect(self.operators[f'P{i}'], state))
                for i in range(3)]
        max_prob_index = np.argmax(probs)
        states_names = ['Samsara', 'Karmic', 'Emptiness']

        return {
            'dominant_state': states_names[max_prob_index],
            'probabilities': probs,
            'certainty': float(max(probs)),
            'note': 'Classification at conventional level (samvrti-satya)'
        }
\end{lstlisting}

\end{appendices}

\begin{thebibliography}{99}
\bibitem{bardo1} Fremantle, F. (2001). \emph{The Tibetan Book of the Dead}. Shambhala Publications.
\bibitem{quantum1} Nielsen, M. A., \& Chuang, I. L. (2010). \emph{Quantum Computation and Quantum Information}. Cambridge University Press.
\bibitem{consciousness1} Hameroff, S., \& Penrose, R. (2014). Consciousness in the universe: A review of the 'Orch OR' theory. \emph{Physics of Life Reviews}, 11(1), 39-78.
\bibitem{buddhism1} Wallace, B. A. (2007). \emph{Contemplative Science: Where Buddhism and Neuroscience Converge}. Columbia University Press.
\bibitem{madhyamaka1} Nāgārjuna. (2013). \emph{The Fundamental Wisdom of the Middle Way: Nāgārjuna's Mūlamadhyamakakārikā}. Oxford University Press.
\bibitem{qutrit1} Lanyon, B. P., et al. (2008). Manipulating biphotonic qutrits. \emph{Physical Review Letters}, 100(6), 060504.
\bibitem{computation1} Tegmark, M. (2000). Importance of quantum decoherence in brain processes. \emph{Physical Review E}, 61(4), 4194.
\bibitem{epistemology1} Varela, F. J., Thompson, E., \& Rosch, E. (2016). \emph{The Embodied Mind: Cognitive Science and Human Experience}. MIT Press.
\end{thebibliography}

\end{document}{EPISTEMOLOGICAL WARNING: #1}
    \vspace{0.1cm}
    \hrule
    \vspace{0.2cm}
}{
    \vspace{0.2cm}
    \hrule
    \end{minipage}
    \end{center}
}

\title{
    \textbf{Quantum Simulation of Consciousness States in the Bardo Thodol:} \\
    \large A Computational Approach via Qutrit Theory and Karmic Dynamics \\
    \normalsize With Epistemological Transparency on Modeling Limits
}

\author{
    \textbf{Horacio Hector Hamann} \\
    \small \href{https://github.com/arathorian/BardoThodol}{https://github.com/arathorian/BardoThodol}
}

\date{November 2025}

\begin{document}

% Enhanced title page
\begin{titlepage}
    \centering
    \vspace*{1cm}

    {\Huge \textbf{Quantum Simulation of the Bardo Thodol}}

    \vspace{0.5cm}
    {\Large Modeling Post-Mortem Consciousness States \\ via Qutrit Systems and Karmic Operators}

    \vspace{0.3cm}
    {\normalsize \textit{With Explicit Epistemological Meta-Modeling}}

    \vspace{1.5cm}

    \begin{figure}[h]
        \centering
        \begin{quantikz}
            \lstick{$\ket{0}$} & \gate{H} & \ctrl{1} & \gate{R_y(\theta)} & \meter{} \\
            \lstick{$\ket{2}$} & \gate{S} & \targ{} & \gate{R_z(\phi)} & \meter{} \\
            \lstick{$\ket{1}$} & \gate{T} & \ctrl{-1} & \gate{H} & \meter{}
        \end{quantikz}
        \caption{Quantum circuit representing transitions between Bardo states}
    \end{figure}

    \vspace{1.5cm}

    {\large \textbf{Horacio Hector Hamann}}

    \vspace{0.3cm}
    {\small \href{https://github.com/arathorian/BardoThodol}{https://github.com/arathorian/BardoThodol}}

    \vspace{1cm}

    \begin{abstract}
        \noindent This article presents an innovative theoretical and computational framework for quantum simulation of consciousness states described in the \emph{Bardo Thodol} (Tibetan Book of the Dead). We propose a qutrit-based model (three-level quantum states) where post-mortem states are represented as quantum superpositions, and karmic transitions as time-evolution operators dependent on attention and karmic accumulation parameters.

        \vspace{0.3cm}
        \noindent Following the Madhyamaka method of Two Truths, this work explicates inherent irreducible paradoxes in mathematical modeling of contemplative phenomena, distinguishing between conventional truth (\emph{saṃvṛti-satya}), ultimate truth (\emph{paramārtha-satya}), and pedagogical use (\emph{upāya}).

        \vspace{0.3cm}
        \noindent \textbf{Keywords:} Bardo Thodol, Quantum Computing, Qutrits, Consciousness States, Simulation, Sunyata, Karma, Quantum Decoherence, Modeling Epistemology
    \end{abstract}

    \vfill
    {\small Project initiated January 2025 · Updated November 2025}
\end{titlepage}

\cleardoublepage
\pagenumbering{roman}
\setcounter{page}{1}

% Table of contents
\tableofcontents

\cleardoublepage
\pagenumbering{arabic}
\setcounter{page}{1}

% Introduction
\section{Introduction: From Sacred Text to Quantum Algorithm}

\subsection{Interdisciplinary Context}

The \emph{Bardo Thodol}, traditionally interpreted as a ritual guide for post-mortem transition in Tibetan tradition, is reformulated in this work as an \textbf{ancestral algorithm} encoding fundamental consciousness state dynamics. This reinterpretation sits at the intersection of:

\begin{itemize}
    \item \textbf{Mahayana Buddhist Philosophy}: Especially the doctrine of emptiness (sunyata) and buddha-nature
    \item \textbf{Quantum Computing}: Multi-state systems and coherence-decoherence dynamics
    \item \textbf{Neurophenomenology}: Scientific study of consciousness states
    \item \textbf{Information Theory}: Processing and transition of informational states
    \item \textbf{Critical Epistemology}: Reflexive analysis of formal modeling limits
\end{itemize}

\subsection{Central Hypothesis}

We formulate our fundamental hypothesis as:

\begin{definition}[Bardo Quantum Simulation Hypothesis]
The Bardo Thodol can be modeled as a multi-state quantum system where:
\begin{equation}
\mathcal{H}_{\text{Bardo}} = \alpha\ket{0} + \beta\ket{1} + \gamma\ket{2}
\end{equation}
with $\ket{0}$ representing manifest reality state (samsara), $\ket{1}$ karmic potential states, and $\ket{2}$ fundamental emptiness (sunyata), where $|\alpha|^2 + |\beta|^2 + |\gamma|^2 = 1$.
\end{definition}

\begin{epistemic_warning}{Truth Level}
This mathematical representation belongs to the conventional level (\emph{saṃvṛti-satya}). At the ultimate level (\emph{paramārtha-satya}), śūnyatā is not a measurable vector state but the empty nature of all phenomena, including the concept of emptiness itself. The model is pedagogical \textbf{upāya} (skillful means), not ontological description.
\end{epistemic_warning}

\subsection{Scientific Justification}

The need for a quantum approach arises from fundamental limitations of classical computational models:

\begin{itemize}
    \item \textbf{Dualism Problem}: Binary systems cannot capture emptiness's non-dual nature
    \item \textbf{Turing Limitations}: Classical machines cannot represent coherent superpositions
    \item \textbf{Probabilistic Nature}: Karmic process is intrinsically probabilistic, not deterministic
\end{itemize}

\subsection{Epistemological Transparency: Modeling Limits}

This work adopts the \textbf{Two Truths} method (Madhyamaka) applied to computational simulation:

\begin{definition}[Truth Levels in the Model]
\begin{enumerate}
    \item \textbf{Conventional Level} (\emph{saṃvṛti-satya}): Quantum mathematics are formally valid in their domain
    \item \textbf{Ultimate Level} (\emph{paramārtha-satya}): Formalism does not capture śūnyatā as ultimate reality
    \item \textbf{Pedagogical Use} (\emph{upāya}): Model is heuristic tool for exploration, not identity with phenomenon
\end{enumerate}
\end{definition}

\subsubsection{Documented Irreducible Paradoxes}

\begin{paradox}[Karmic Quantification]
\label{paradox:karma}
Assigning numerical values to karma (e.g., $k_{\text{clarity}}=0.8$, $k_{\text{attachment}}=0.3$) \textbf{reifies} what Abhidharma describes as impermanent flux (\emph{anitya}) without fixed substance (\emph{anātman}).

\textbf{Irreducible gap:} Karma in Madhyamaka lacks \emph{svabhāva} (inherent nature), being a process of interdependent origination (\emph{pratītyasamutpāda}), not measurable magnitude.

\textbf{Pedagogical value:} Parameters allow exploring how different habitual tendencies affect transitions, without affirming karma \emph{is} these numbers. Heuristic function, not descriptive.
\end{paradox}

\begin{paradox}[Emptiness Reification]
\label{paradox:sunyata}
Representing śūnyatā as vector $\ket{2} = [0,0,1]^T$ in Hilbert space contradicts its nature as \emph{niḥsvabhāva} (absence of inherent being).

\textbf{Irreducible gap:} Converting emptiness into a separate mathematical state is exactly the type of reification (\emph{saṃjñā}) that Prajñāpāramitā warns against. It is irreducible performative contradiction.

\textbf{Pedagogical value:} Demonstrates need for non-binary frameworks transcending classical logic. $\ket{2}$ does not \emph{is} emptiness, it \emph{points toward} it like finger pointing at moon.
\end{paradox}

\begin{paradox}[Model Temporality]
\label{paradox:time}
Temporal evolution $\unitary(t) = e^{-i\hamiltonian t}$ requires time as continuous parameter, while in deep meditative states (\emph{samādhi}), temporal experience collapses.

\textbf{Irreducible gap:} Mathematical formalism cannot model atemporal experience without structural self-contradiction. \emph{Kāla} (time) is mental construction, not absolute.

\textbf{Pedagogical value:} Shows transition dynamics as sequential process useful for conceptual understanding. User must remember mathematical time is model artifact.
\end{paradox}

\begin{table}[h]
\centering
\caption{Meta-modeling: Conventional Truth vs Ultimate Truth}
\label{tab:two_truths}
\begin{tabular}{p{3cm}p{5.5cm}p{5.5cm}}
\toprule
\textbf{Aspect} & \textbf{Saṃvṛti (Conventional)} & \textbf{Paramārtha (Ultimate)} \\
\midrule
Emptiness & Vector $\ket{2} = [0,0,1]^T$ & \emph{Niḥsvabhāva} without substance \\
Karma & Operator $\karmaop$ with numerical parameters & \emph{Pratītyasamutpāda} non-quantifiable \\
Time & Parameter $t \in \mathbb{R}$ & Mental construction (\emph{kāla}) \\
Measurement & Collapse $\ket{\psi} \to \ket{i}$ & Non-dual \emph{rigpa} without observer \\
Utility & Formally valid & Tool (\emph{upāya}) \\
\bottomrule
\end{tabular}
\end{table}

% Theoretical Framework
\section{Theoretical Framework: Quantum and Philosophical Foundations}

\subsection{Qutrit System for Consciousness States}

We define our three-dimensional Hilbert space to model fundamental states:

\begin{equation}
\mathcal{H} = \text{span}\{\ket{0}, \ket{1}, \ket{2}\}
\end{equation}

With corresponding projection operators:

\begin{equation}
P_i = \ket{i}\bra{i}, \quad i \in \{0,1,2\}
\end{equation}

\begin{definition}[Fundamental States - Conventional Level]
\begin{align*}
\ket{0} &= \begin{bmatrix} 1 \\ 0 \\ 0 \end{bmatrix} \quad \text{(Manifest reality - Samsara)} \\
\ket{1} &= \begin{bmatrix} 0 \\ 1 \\ 0 \end{bmatrix} \quad \text{(Karmic potential - Latent states)} \\
\ket{2} &= \begin{bmatrix} 0 \\ 0 \\ 1 \end{bmatrix} \quad \text{(Points toward Sunyata)}
\end{align*}
\end{definition}

\begin{epistemic_warning}{Base State Interpretation}
The three basis vectors are NOT ontologically separate realities. At the ultimate level, all states interpenetrate without fixed boundary. Mathematical separation is pedagogical convention for formal analysis.
\end{epistemic_warning}

\subsection{Karmic Hamiltonian and Evolution Operators}

The evolution operator incorporates karmic and attention parameters:

\begin{equation}
\hamiltonian_K = \sum_{i\neq j} k_{ij}(\ket{i}\bra{j} + \ket{j}\bra{i}) + \sum_i \epsilon_i \ket{i}\bra{i}
\end{equation}

where $k_{ij}$ represents karmic couplings between states (subject to Paradox~\ref{paradox:karma}) and $\epsilon_i$ intrinsic potentials of each state.

\subsection{Six Bardos as Quantum Transitions}

We model the six Bardo states as quantum transition sequences:

\begin{enumerate}
    \item \textbf{Moment of Death Bardo (Chikhai Bardo)}: $\ket{2} \otimes \ket{k}$
    \item \textbf{Reality Bardo (Chonyid Bardo)}: $\sum_k c_k\ket{k}$
    \item \textbf{Becoming Bardo (Sidpa Bardo)}: $\ket{0} \leftarrow$ Measurement
\end{enumerate}

\subsection{Conceptual Genesis: From ERROR 505 to Quantum Qutrit}

The conceptual turning point arose from analyzing anthropomorphic digital classifications applied to post-mortem consciousness states. Identifying "ERROR 505" as "Deity recognition failure" revealed fundamental limitation in classical computational models.

\subsubsection{Binary Paradigm Limitation}

Interpretation as "error" emerged from binary framework unable to represent:
\begin{itemize}
    \item Uncollapsed quantum superposition states
    \item Emptiness (śūnyatā) as fundamental state
    \item Non-actualized karmic potentiality
\end{itemize}

\subsubsection{Transition to Quantum Model}

Resolution required transcending Boolean logic via:
\begin{equation}
\mathcal{H}_{\text{Bardo}} = \alpha\ket{0} + \beta\ket{1} + \gamma\ket{2}
\end{equation}
where $\ket{2}$ points toward fundamental emptiness, not error state.

This paradigmatic transition enabled reinterpreting "errors" as windows to maximum quantum potentiality states where karma can be reprogrammed.

% Methodology
\section{Methodology: Computational Implementation}

\subsection{Simulation System Architecture}

We implement the system using Python 3.11 with the following main libraries:

\begin{lstlisting}[caption={Quantum simulation environment configuration}, label={lst:quantum_setup}]
import numpy as np
import qutip as qt
from scipy.linalg import expm
import matplotlib.pyplot as plt
from mpl_toolkits.mplot3d import Axes3D
import seaborn as sns

class BardoQuantumSystem:
    """Quantum system with epistemological reflexivity"""

    def __init__(self, karma_params=None):
        self.karma_params = karma_params or {
            'clarity': 0.8, 'attachment': 0.3,
            'compassion': 0.9, 'wisdom': 0.7
        }
        self.dim = 3
        self.operators = self._create_operators()
        self.current_state = qt.basis(self.dim, 2)

        # Document model limitations
        self.model_limitations = {
            'karma_reification':
                'Numerical parameters reify impermanent flux',
            'temporal_assumption':
                'Time t is convention, not ultimate reality',
            'measurement_duality':
                'Maintains observer-observed framework'
        }

    def _create_operators(self):
        """Creates fundamental quantum operators"""
        # Basis states
        kets = [qt.basis(3, i) for i in range(3)]

        # Projectors P0, P1, P2
        P = {f'P{i}': kets[i] * kets[i].dag() for i in range(3)}

        # Transition operators
        S01 = kets[0] * kets[1].dag()
        S12 = kets[1] * kets[2].dag()
        S20 = kets[2] * kets[0].dag()

        # Base Hamiltonian
        H0 = 0.1*P['P0'] + 0.2*P['P1'] + 0.3*P['P2']

        # Karmic operator (subject to Paradox 1)
        K = (self.karma_params['attachment'] * (S01 + S01.dag()) +
             self.karma_params['clarity'] * (S12 + S12.dag()) +
             self.karma_params['compassion'] * (S20 + S20.dag()))

        P.update({'S01': S01, 'S12': S12, 'S20': S20, 'H0': H0, 'K': K})
        return P
\end{lstlisting}

\subsection{Time Evolution Algorithm}

The main algorithm simulates complete evolution through Bardo states:

\begin{lstlisting}[caption={Bardo evolution algorithm}, label={lst:bardo_evolution}]
def simulate_bardo_transition(self, time_steps=1000,
                            attention_function='logistic'):
    """Simulates transition with assumption documentation"""

    times = np.linspace(0, 4*np.pi, time_steps)
    results = {
        'probabilities': [],
        'coherence': [],
        'purity': [],
        'states': [],
        'epistemic_notes': []
    }

    current_state = self.current_state

    for t in times:
        # Attention factor (temporal convention)
        attention = self._attention_evolution(t, attention_function)

        # Effective Hamiltonian
        H_eff = self.operators['H0'] + attention * self.operators['K']

        # Incremental unitary evolution
        dt = times[1] - times[0] if len(times) > 1 else 0.01
        U = (-1j * dt * H_eff).expm()
        evolved_state = U * current_state
        current_state = evolved_state

        # Calculate probabilities using projectors
        probs = [qt.expect(self.operators[f'P{i}'], evolved_state)
                for i in range(self.dim)]

        coherence = self._calculate_coherence(evolved_state)
        purity = self._calculate_purity(evolved_state)

        results['probabilities'].append(probs)
        results['coherence'].append(coherence)
        results['purity'].append(purity)
        results['states'].append(evolved_state)

        # Epistemological note every 100 steps
        if len(results['states']) % 100 == 0:
            note = self._generate_epistemic_note(evolved_state, t)
            results['epistemic_notes'].append(note)

        current_state = evolved_state

    return results, times

def _generate_epistemic_note(self, state, time):
    """Generates note on model limits at this point"""
    probs = [qt.expect(self.operators[f'P{i}'], state)
             for i in range(3)]
    dominant = np.argmax(probs)

    notes = {
        0: f"t={time:.2f}: High P(|0>) signals manifestation, "
           f"but form is empty",
        1: f"t={time:.2f}: High P(|1>) indicates potential, "
           f"not substantial karma",
        2: f"t={time:.2f}: High P(|2>) points to sunyata, "
           f"does not describe it"
    }
    return notes[dominant]
\end{lstlisting}

% Results
\section{Results and Simulations}

\subsection{Temporal Probability Evolution}

\begin{figure}[h]
    \centering
    \includegraphics[width=0.9\textwidth]{figures/state_evolution.pdf}
    \caption{
        Temporal evolution of probabilities and quantum metrics in Bardo system (Conventional Level).
        (A) Fundamental state probabilities: Samsara ($\ket{0}$), Karmic Potential ($\ket{1}$) and Emptiness pointer ($\ket{2}$).
        (B) Quantum coherence and state purity evolution, showing periods of coherent superposition and decoherence.
        \textbf{Epistemological note:} These trajectories are formally valid but do not describe direct contemplative experience.
    }
    \label{fig:state_evolution}
\end{figure}

\subsection{Quantum Coherence Analysis}

Quantum coherence is maintained during Bardo transitions, with characteristic patterns:

\begin{equation}
C(\rho) = \sum_{i\neq j} |\rho_{ij}|
\end{equation}

\begin{epistemic_warning}{Coherence Interpretation}
Mathematical quantum coherence is ANALOGOUS (not identical) to phenomenological "non-dual interpenetration". Number $C(\rho)$ does not directly measure contemplative clarity (\emph{prajñā}), but points toward it as formal correlate.
\end{epistemic_warning}

\begin{table}[h]
\centering
\caption{Coherence metrics by Bardo state (Conventional Level)}
\begin{tabular}{lccc}
\toprule
\textbf{Bardo State} & \textbf{Coherence} & \textbf{Purity} & \textbf{Entropy} \\
\midrule
Chikhai Bardo & 0.95 $\pm$ 0.02 & 0.98 $\pm$ 0.01 & 0.12 $\pm$ 0.03 \\
Chonyid Bardo & 0.87 $\pm$ 0.04 & 0.92 $\pm$ 0.03 & 0.28 $\pm$ 0.05 \\
Sidpa Bardo & 0.45 $\pm$ 0.07 & 0.78 $\pm$ 0.06 & 0.65 $\pm$ 0.08 \\
\bottomrule
\end{tabular}
\end{table}

% Discussion
\section{Discussion: Interdisciplinary Implications}

\subsection{Central Hypothesis Validation}

Our results demonstrate that (at conventional level):

\begin{enumerate}
    \item Qutrit model can effectively represent emptiness non-duality \textit{as logical structure}
    \item Transitions between Bardo states follow coherent quantum dynamics \textit{as formal analogy}
    \item Metaphorical "ERROR 505" corresponds mathematically to uncollapsed superposition states \textit{pointing toward potentiality}
\end{enumerate}

\begin{epistemic_warning}{Validation Scope}
"Validation" is internal to mathematical model. We do not affirm Bardo Thodol "is" quantum algorithm, but that quantum formalism can be used as \emph{upāya} to explore its logical structure. Post-mortem contemplative experience remains outside model scope.
\end{epistemic_warning}

\subsection{Comparison with Classical Models}

\begin{table}[h]
\centering
\caption{Comparison between classical and quantum models}
\begin{tabular}{lcc}
\toprule
\textbf{Feature} & \textbf{Classical Model} & \textbf{Quantum Model} \\
\midrule
Emptiness representation & ERROR 505 & Pointer state $\ket{2}$ \\
Superposed states & Not possible & Fundamental \\
Probabilistic nature & Simulated & Intrinsic \\
Non-local transitions & No & Yes (analogically) \\
Temporal coherence & No & Yes \\
\textbf{Documented paradoxes} & \textbf{Ignored} & \textbf{Explicit} \\
\bottomrule
\end{tabular}
\end{table}

\subsection{Implications for Consciousness Science}

Our work suggests (as exploratory hypothesis, not ontological claim):

\begin{itemize}
    \item Consciousness states might follow quantum dynamics \textit{in certain structural aspects}
    \item Deep meditation might affect quantum coherence parameters \textit{measurable neurophysiologically}
    \item Computational consciousness models must consider quantum frameworks \textit{with explicit limit documentation}
\end{itemize}

\subsection{Recognized Project Limitations}

\begin{enumerate}
    \item \textbf{Phenomenological gap}: Model does not capture direct experience (\emph{pratyakṣa}) of bardo states
    \item \textbf{Parametric reductionism}: Quantified karma contradicts its interdependent process nature
    \item \textbf{Artificial temporality}: Mathematical time does not reflect \emph{samādhi} timelessness
    \item \textbf{Observational dualism}: Maintains measurer-measured separation absent in \emph{rigpa}
    \item \textbf{Emptiness reification}: $\ket{2}$ as vector contradicts \emph{niḥsvabhāva}
\end{enumerate}

These limitations are not "problems to solve" but inherent features of mathematical modeling of contemplative phenomena.

% Conclusion
\section{Conclusion and Future Work}

\subsection{Main Conclusions}

\begin{enumerate}
    \item We demonstrated feasibility of \textit{structurally} modeling Bardo Thodol consciousness states using quantum systems
    \item Qutrit approach overcomes binary model limitations \textit{at formal logic level}
    \item Emptiness (sunyata) finds natural mathematical representation in quantum superpositions \textit{as analogy, not identity}
    \item Karmic dynamics can be implemented as quantum operators \textit{with explicit reification recognition}
    \item \textbf{Explicit irreducible paradox documentation} converts project into reflexive meta-model
\end{enumerate}

\subsection{Methodological Framework: Three Truths Applied}

\begin{table}[h]
\centering
\caption{Two Truths method application to modeling}
\begin{tabular}{p{3.5cm}p{5.5cm}p{5cm}}
\toprule
\textbf