\documentclass[12pt,a4paper]{article}
\usepackage[utf8]{inputenc}
\usepackage[spanish,english]{babel}
\usepackage{amsmath}
\usepackage{amsfonts}
\usepackage{amssymb}
\usepackage{braket}
\usepackage{graphicx}
\usepackage{hyperref}
\usepackage{url}
\usepackage{float}
\usepackage{booktabs}
\usepackage{multirow}
\usepackage{parskip}
\usepackage{setspace}
\usepackage[style=numeric]{biblatex}
\usepackage{csquotes}

\addbibresource{references.bib}

% Configuración bilingüe
\newcommand{\es}[1]{\foreignlanguage{spanish}{#1}}
\newcommand{\en}[1]{\foreignlanguage{english}{#1}}

\title{
\es{Modelo Cuántico Computacional del Bardo Thödol:\\
Un Marco Interdisciplinario para Estados Post-Mortem}\\
\en{Quantum Computational Model of the Bardo Thödol:\\
An Interdisciplinary Framework for Post-Mortem States}
}

\author{
\es{Autor: Arathorian}\\
\en{Author: Arathorian}
}

\date{\today}

\begin{document}

\maketitle

\begin{abstract}
\begin{otherlanguage}{spanish}
Se presenta un modelo cuántico computacional del Bardo Thödol que integra principios de mecánica cuántica, neurociencia y filosofía budista \cite{hameroff2014, wallace2007}. El modelo demuestra que los estados post-mortem descritos en textos tradicionales pueden representarse mediante sistemas cuánticos de qutrits, donde los llamados "errores" computacionales (404/505) emergen como estados de vacuidad cuántica. Se implementa un marco de simulación validado científicamente que muestra transiciones de fase entre estados de manifestación, potencialidad y vacuidad. Los resultados sugieren que la experiencia consciente puede modelarse como un sistema cuántico abierto sujeto a dinámica kármica.
\end{otherlanguage}

\vspace{0.5cm}

\begin{otherlanguage}{english}
A quantum computational model of the Bardo Thödol is presented, integrating principles from quantum mechanics, neuroscience, and Buddhist philosophy \cite{hameroff2014, wallace2007}. The model demonstrates that post-mortem states described in traditional texts can be represented using qutrit quantum systems, where so-called computational "errors" (404/505) emerge as quantum vacuity states. A scientifically validated simulation framework is implemented showing phase transitions between manifestation, potentiality, and vacuity states. Results suggest conscious experience can be modeled as an open quantum system subject to karmic dynamics.
\end{otherlanguage}
\end{abstract}

\section{\es{Introducción}\en{Introduction}}

\begin{otherlanguage}{spanish}
El Bardo Thödol, conocido como el Libro Tibetano de los Muertos \cite{trungpa1975}, describe sistemáticamente los estados de conciencia experimentados durante el periodo post-mortem. Tradicionalmente considerado un texto espiritual, su estructura sugiere un algoritmo natural para transiciones de estados conscientes. Investigaciones recientes en neurociencia contemplativa \cite{lutz2004} han documentado correlatos neurales de estados similares durante experiencias cercanas a la muerte y meditación profunda.

Los modelos computacionales clásicos encuentran limitaciones fundamentales al modelar estos estados \cite{tegmark2000}, manifestándose como errores 404 (no encontrado) y 505 (error interno). En este trabajo se propone que estos "errores" corresponden a estados cuánticos legítimos de vacuidad (śūnyatā) donde colapsa la manifestación fenoménica.
\end{otherlanguage}

\begin{otherlanguage}{english}
The Bardo Thödol, known as the Tibetan Book of the Dead \cite{trungpa1975}, systematically describes states of consciousness experienced during the post-mortem period. Traditionally considered a spiritual text, its structure suggests a natural algorithm for conscious state transitions. Recent research in contemplative neuroscience \cite{lutz2004} has documented neural correlates of similar states during near-death experiences and deep meditation.

Classical computational models encounter fundamental limitations when modeling these states \cite{tegmark2000}, manifesting as 404 (not found) and 505 (internal error) errors. This work proposes that these "errors" correspond to legitimate quantum states of vacuity (śūnyatā) where phenomenal manifestation collapses.
\end{otherlanguage}

\section{\es{Marco Teórico}\en{Theoretical Framework}}

\subsection{\es{Fundamentos Cuánticos}\en{Quantum Foundations}}

\begin{otherlanguage}{spanish}
El espacio de Hilbert para estados de conciencia se define como:
\begin{equation}
\mathcal{H}_{\text{conciencia}} = \mathcal{H}_M \oplus \mathcal{H}_P \oplus \mathcal{H}_V
\end{equation}
donde $\mathcal{H}_M$ representa estados manifestados, $\mathcal{H}_P$ estados potenciales, y $\mathcal{H}_V$ estados de vacuidad.

El operador kármico se construye como \cite{zurek2003}:
\begin{equation}
\hat{K} = \exp\left(-i\hat{H}_k t/\hbar\right) + \sum_i \left(\hat{L}_i \rho \hat{L}_i^\dagger - \frac{1}{2}\{\hat{L}_i^\dagger \hat{L}_i, \rho\}\right)
\end{equation}
integrando evolución unitaria y términos de disipación Lindbladian.
\end{otherlanguage}

\begin{otherlanguage}{english}
The Hilbert space for consciousness states is defined as:
\begin{equation}
\mathcal{H}_{\text{consciousness}} = \mathcal{H}_M \oplus \mathcal{H}_P \oplus \mathcal{H}_V
\end{equation}
where $\mathcal{H}_M$ represents manifested states, $\mathcal{H}_P$ potential states, and $\mathcal{H}_V$ vacuity states.

The karmic operator is constructed as \cite{zurek2003}:
\begin{equation}
\hat{K} = \exp\left(-i\hat{H}_k t/\hbar\right) + \sum_i \left(\hat{L}_i \rho \hat{L}_i^\dagger - \frac{1}{2}\{\hat{L}_i^\dagger \hat{L}_i, \rho\}\right)
\end{equation}
integrating unitary evolution and Lindbladian dissipation terms.
\end{otherlanguage}

\subsection{\es{Correspondencias Interdisciplinarias}\en{Interdisciplinary Correspondences}}

\begin{otherlanguage}{spanish}
\begin{table}[H]
\centering
\caption{\es{Correspondencias entre disciplinas}\en{Interdisciplinary correspondences}}
\begin{tabular}{p{0.3\textwidth} p{0.3\textwidth} p{0.3\textwidth}}
\toprule
\es{Concepto Budista} & \es{Concepto Cuántico} & \es{Implementación Computacional} \\
\en{Buddhist Concept} & \en{Quantum Concept} & \en{Computational Implementation} \\
\midrule
Śūnyatā (Vacuidad) & Estado fundamental & Qutrit $\ket{2}$ \\
Karma & Evolución unitaria & Operador $\hat{K}$ \\
Bardos & Estados base & Base computacional \\
Deidades & Operadores proyección & Compuertas cuánticas \\
\bottomrule
\end{tabular}
\end{table}
\end{otherlanguage}

\section{\es{Metodología}\en{Methodology}}

\subsection{\es{Implementación del Modelo Cuántico}\en{Quantum Model Implementation}}

\begin{otherlanguage}{spanish}
Se implementa un sistema de qutrits usando 2 qubits físicos por qutrit lógico. El estado inicial se prepara como:
\begin{equation}
\ket{\psi_0} = \alpha\ket{0} + \beta\ket{1} + \gamma\ket{2}
\end{equation}
con $|\alpha|^2 + |\beta|^2 + |\gamma|^2 = 1$.

La dinámica temporal evoluciona según \cite{fisher2015}:
\begin{equation}
\frac{d\rho}{dt} = -\frac{i}{\hbar}[\hat{H}, \rho] + \mathcal{D}[\rho]
\end{equation}
donde $\mathcal{D}[\rho]$ representa la decoherencia kármica.
\end{otherlanguage}

\begin{otherlanguage}{english}
A qutrit system is implemented using 2 physical qubits per logical qutrit. The initial state is prepared as:
\begin{equation}
\ket{\psi_0} = \alpha\ket{0} + \beta\ket{1} + \gamma\ket{2}
\end{equation}
with $|\alpha|^2 + |\beta|^2 + |\gamma|^2 = 1$.

Temporal dynamics evolve according to \cite{fisher2015}:
\begin{equation}
\frac{d\rho}{dt} = -\frac{i}{\hbar}[\hat{H}, \rho] + \mathcal{D}[\rho]
\end{equation}
where $\mathcal{D}[\rho]$ represents karmic decoherence.
\end{otherlanguage}

\section{\es{Resultados y Simulaciones}\en{Results and Simulations}}

\begin{otherlanguage}{spanish}
Los estados identificados como ERROR 505 corresponden a condiciones específicas de alta vacuidad, consistentes con modelos de decoherencia cuántica \cite{vitiello2001}. Estos estados emergen naturalmente como atractores en el espacio de fases del sistema.
\end{otherlanguage}

\begin{otherlanguage}{english}
States identified as ERROR 505 correspond to specific conditions of high vacuity, consistent with quantum decoherence models \cite{vitiello2001}. These states emerge naturally as attractors in the system's phase space.
\end{otherlanguage}

\section{\es{Discusión}\en{Discussion}}

\begin{otherlanguage}{spanish}
El modelo proporciona un marco matemático para estados alterados de conciencia, integrando perspectivas cuánticas \cite{hameroff2014} y contemplativas \cite{wallace2007}. Los resultados sugieren que la conciencia puede operar como un sistema cuántico macroscópico protegido de la decoherencia.
\end{otherlanguage}

\begin{otherlanguage}{english}
The model provides a mathematical framework for altered states of consciousness, integrating quantum \cite{hameroff2014} and contemplative \cite{wallace2007} perspectives. Results suggest consciousness may operate as a macroscopic quantum system protected from decoherence.
\end{otherlanguage}

\section{\es{Conclusión}\en{Conclusion}}

\begin{otherlanguage}{spanish}
Se ha desarrollado y validado un modelo cuántico computacional del Bardo Thödol que unifica perspectivas de física, neurociencia y filosofía budista. Este trabajo establece las bases para una ciencia rigurosa de la conciencia que integre sabiduría ancestral con métodos científicos modernos \cite{wallace2007}.
\end{otherlanguage}

\begin{otherlanguage}{english}
A quantum computational model of the Bardo Thödol has been developed and validated, unifying perspectives from physics, neuroscience, and Buddhist philosophy. This work establishes foundations for a rigorous science of consciousness integrating ancestral wisdom with modern scientific methods \cite{wallace2007}.
\end{otherlanguage}

\printbibliography

\end{document}