\documentclass[12pt, a4paper]{article}

\usepackage[spanish,es-noshorthands]{babel}
\usepackage[utf8]{inputenc}
\usepackage[T1]{fontenc}
\usepackage{amsmath, amssymb, amsthm, mathrsfs}
\usepackage{physics}
\usepackage{microtype}
\usepackage{siunitx}
\sisetup{
    output-decimal-marker = {,},
    group-separator = {.},
    per-mode = symbol
}

\usepackage{graphicx, wrapfig, booktabs, multirow, multicol, xcolor}
\usepackage{geometry, abstract, caption, subcaption}
\usepackage{listings}
\usepackage[hyphens]{url}
\usepackage{hyperref}
\usepackage{appendix}
\hypersetup{
    colorlinks=true,
    linkcolor=blue,
    filecolor=magenta,
    urlcolor=cyan,
    pdftitle={Simulación Cuántica del Bardo Thodol},
    pdfauthor={Horacio Hector Hamann},
    pdfsubject={Computación Cuántica y Filosofía Budista},
    pdfkeywords={Bardo Thodol, Computación Cuántica, Qutrits, Conciencia, Simulación},
    breaklinks=true,
    pdfencoding=auto
}
\usepackage{fancyhdr}

% Quantikz después de physics
\usepackage{tikz}
\usetikzlibrary{quantikz}

% Configuración de geometría
\geometry{left=2.5cm, right=2.5cm, top=2.5cm, bottom=2.5cm}

% SOLUCIÓN COMPLETA: fancyhdr + contador de páginas
\setlength{\headheight}{14.5pt}

% Configuración de hyperref
\hypersetup{
    colorlinks=true,
    linkcolor=blue,
    filecolor=magenta,
    urlcolor=cyan,
    pdftitle={Simulacion Cuantica del Bardo Thodol},
    pdfauthor={Horacio Hector Hamann},
    pdfsubject={Computacion Cuantica y Filosofia Budista},
    pdfkeywords={Bardo Thodol, Computacion Cuantica, Qutrits, Conciencia, Simulacion},
    bookmarksnumbered=true,
    bookmarksopen=true,
    pdfstartview=FitH
}

% Colores personalizados
\definecolor{deepblue}{rgb}{0,0.2,0.6}
\definecolor{deepred}{rgb}{0.6,0,0}
\definecolor{deepgreen}{rgb}{0,0.5,0}
\definecolor{codegray}{rgb}{0.5,0.5,0.5}

% Configuración definitiva de listings
\lstset{
    language=Python,
    basicstyle=\ttfamily\small,
    keywordstyle=\color{deepblue},
    commentstyle=\color{deepgreen},
    stringstyle=\color{deepred},
    numbers=left,
    numberstyle=\tiny\color{codegray},
    stepnumber=1,
    numbersep=5pt,
    backgroundcolor=\color{white},
    showspaces=false,
    showstringspaces=false,
    showtabs=false,
    tabsize=4,
    captionpos=b,
    breaklines=true,
    breakatwhitespace=false,
    escapeinside={\%*}{*)},
    inputencoding=utf8,
    extendedchars=true,
    literate={á}{{\'a}}1 {é}{{\'e}}1 {í}{{\'i}}1 {ó}{{\'o}}1 {ú}{{\'u}}1
             {Á}{{\'A}}1 {É}{{\'E}}1 {Í}{{\'I}}1 {Ó}{{\'O}}1 {Ú}{{\'U}}1
             {ñ}{{\~n}}1 {Ñ}{{\~N}}1
}

% Configuración de encabezados
\pagestyle{fancy}
\fancyhf{}
\fancyhead[L]{\small Simulacion Cuantica del Bardo Thodol}
\fancyhead[R]{\small \thepage}
\renewcommand{\headrulewidth}{0.4pt}

% COMANDOS:
\newcommand{\expected}[1]{\langle #1 \rangle}
\newcommand{\state}[1]{\ket{\psi_{#1}}}
\newcommand{\hamiltonian}{\hat{H}}
\newcommand{\unitary}{\hat{U}}
\newcommand{\karmaop}{\hat{K}}
\newcommand{\vacuumstate}{\ket{2}}
\newcommand{\manifeststate}{\ket{0}}
\newcommand{\potentialstate}{\ket{1}}

% Teoremas y definiciones
\newtheorem{teorema}{Teorema}
\newtheorem{definicion}{Definicion}
\newtheorem{corolario}{Corolario}
\newtheorem{lema}{Lema}

\title{
    \textbf{Simulacion Cuantica de los Estados de Conciencia del Bardo Thodol:} \\
    \large Un Enfoque Computacional desde la Teoria de Qutrits y Dinamica Karmica
}

\author{
    \textbf{Horacio Hector Hamann} \\
    \small \href{https://github.com/arathorian/BardoThodol}{https://github.com/arathorian/BardoThodol}
}

\date{Julio 2025}

\begin{document}

% Portada mejorada
\begin{titlepage}
    \centering
    \vspace*{1cm}
    
    {\Huge \textbf{Simulacion Cuantica del Bardo Thodol}}
    
    \vspace{0.5cm}
    {\Large Modelado de Estados de Conciencia Post-Mortem \\ mediante Sistemas de Qutrits y Operadores Karmicos}
    
    \vspace{1.5cm}
    
    \begin{figure}[h]
        \centering
        \begin{quantikz}
            \lstick{$\ket{0}$} & \gate{H} & \ctrl{1} & \gate{R_y(\theta)} & \meter{} \\
            \lstick{$\ket{2}$} & \gate{S} & \targ{} & \gate{R_z(\phi)} & \meter{} \\
            \lstick{$\ket{1}$} & \gate{T} & \ctrl{-1} & \gate{H} & \meter{}
        \end{quantikz}
        \caption{Circuito cuantico representando transiciones entre estados del Bardo}
    \end{figure}

    \vspace{1.5cm}

    {\large \textbf{Horacio Hector Hamann}}

    \vspace{0.3cm}
    {\small \href{https://github.com/arathorian/BardoThodol}{https://github.com/arathorian/BardoThodol}}

    \vspace{1cm}

    \begin{abstract}
        \noindent Este articulo presenta un marco teorico y computacional innovador para la simulacion cuantica de los estados de conciencia descritos en el \emph{Bardo Thodol} (Libro Tibetano de los Muertos). Proponemos un modelo basado en sistemas de qutrits (estados cuanticos de tres niveles) donde los estados post-mortem son representados como superposiciones cuanticas, y las transiciones karmicas como operadores de evolucion temporal dependientes de parametros de atencion y acumulaciones karmicas.

        \vspace{0.3cm}
        \noindent Demostramos que la logica ternaria cuantica supera fundamentalmente las limitaciones de los modelos binarios clasicos para representar la no-dualidad de la vacuidad (sunyata), reinterpretando el estado de "ERROR 505" metaforico como superposicion cuantica no colapsada $\ket{2}$. 

        \vspace{0.3cm}
        \noindent \textbf{Palabras clave:} Bardo Thodol, Computacion Cuantica, Qutrits, Estados de Conciencia, Simulacion, Sunyata, Karma, Decoherencia Cuantica
    \end{abstract}

    \vfill
    {\small Proyecto iniciado en Enero 2025 · Publicado en Julio 2025 . Actualizado en Noviembre 2025 }
\end{titlepage}

% SOLUCIÓN FINAL: Limpiar página correctamente
\cleardoublepage
\pagenumbering{arabic}
\setcounter{page}{1}

% Índice
\tableofcontents
\clearpage
\pagenumbering{arabic}
\setcounter{page}{1}

% Introducción
\section{Introduccion: Del Texto Sagrado al Algoritmo Cuantico}

\subsection{Contexto Interdisciplinario}

El \emph{Bardo Thodol}, tradicionalmente interpretado como una guia ritual para la transicion post-mortem en la tradicion tibetana, es reformulado en este trabajo como un \textbf{algoritmo ancestral} que codifica la dinamica fundamental de estados de conciencia. Esta reinterpretacion se situa en la interseccion de:

\begin{itemize}
    \item \textbf{Filosofia Budista Mahayana}: Especialmente la doctrina de la vacuidad (sunyata) y la naturaleza budica
    \item \textbf{Computacion Cuantica}: Sistemas de multiples estados y dinamicas de coherencia-decoherencia
    \item \textbf{Neurofenomenologia}: Estudio cientifico de los estados de conciencia
    \item \textbf{Teoria de la Informacion}: Procesamiento y transicion de estados informacionales
\end{itemize}

\subsection{Hipotesis Central}

Formulamos nuestra hipotesis fundamental como:

\begin{definicion}[Hipotesis de Simulacion Cuantica del Bardo]
El Bardo Thodol puede ser modelado como un sistema cuantico de multiples estados donde:
\begin{equation}
\mathcal{H}_{\text{Bardo}} = \alpha\ket{0} + \beta\ket{1} + \gamma\ket{2}
\end{equation}
con $\ket{0}$ representando el estado de realidad manifiesta (samsara), $\ket{1}$ estados potenciales karmicos, y $\ket{2}$ la vacuidad fundamental (sunyata), donde $|\alpha|^2 + |\beta|^2 + |\gamma|^2 = 1$.
\end{definicion}

\subsection{Justificacion Cientifica}

La necesidad de un enfoque cuantico surge de las limitaciones fundamentales de los modelos computacionales clasicos:

\begin{itemize}
    \item \textbf{Problema del Dualismo}: Los sistemas binarios no pueden capturar la naturaleza no-dual de la vacuidad
    \item \textbf{Limitaciones de Turing}: La maquina clasica no puede representar superposiciones coherentes
    \item \textbf{Naturaleza Probabilistica}: El proceso karmico es intrinsecamente probabilistico, no determinista
\end{itemize}

% Marco Teórico
\section{Marco Teorico: Fundamentos Cuanticos y Filosoficos}

\subsection{Sistema de Qutrits para Estados de Conciencia}

Definimos nuestro espacio de Hilbert tridimensional para modelar los estados fundamentales:

\begin{equation}
\mathcal{H} = \text{span}\{\ket{0}, \ket{1}, \ket{2}\}
\end{equation}

Con los operadores de proyeccion correspondientes:

\begin{equation}
P_i = \ket{i}\bra{i}, \quad i \in \{0,1,2\}
\end{equation}

\begin{definicion}[Estados Fundamentales]
\begin{align*}
\ket{0} &= \begin{bmatrix} 1 \\ 0 \\ 0 \end{bmatrix} \quad \text{(Realidad manifiesta - Samsara)} \\
\ket{1} &= \begin{bmatrix} 0 \\ 1 \\ 0 \end{bmatrix} \quad \text{(Potencial karmico - Estados latentes)} \\
\ket{2} &= \begin{bmatrix} 0 \\ 0 \\ 1 \end{bmatrix} \quad \text{(Vacuidad fundamental - Sunyata)}
\end{align*}
\end{definicion}

\subsection{Hamiltoniano Karmico y Operadores de Evolucion}

El operador de evolucion incorpora parametros karmicos y de atencion:

\begin{equation}
\hamiltonian_K = \sum_{i\neq j} k_{ij}(\ket{i}\bra{j} + \ket{j}\bra{i}) + \sum_i \epsilon_i \ket{i}\bra{i}
\end{equation}

donde $k_{ij}$ representa los acoplamientos karmicos entre estados y $\epsilon_i$ los potenciales intrinsecos de cada estado.

\subsection{Los Seis Bardos como Transiciones Cuanticas}

Modelamos los seis estados del Bardo como secuencias de transiciones cuanticas:

\begin{enumerate}
    \item \textbf{Bardo del Momento de la Muerte (Chikhai Bardo)}: $\ket{2} \otimes \ket{k}$
    \item \textbf{Bardo de la Realidad (Chonyid Bardo)}: $\sum_k c_k\ket{k}$
    \item \textbf{Bardo del Devenir (Sidpa Bardo)}: $\ket{0} \leftarrow$ Medida
\end{enumerate}

\subsection{Genesis Conceptual: Del ERROR 505 al Qutrit Cuantico}

El punto de inflexion conceptual surgio del analisis de las clasificaciones digitales antropomorficas aplicadas a estados de conciencia post-mortem. La identificacion de \"ERROR 505\" como \"Falta de reconocimiento de deidad\" revelaba una limitacion fundamental en los modelos computacionales clasicos.

\subsubsection{Limitacion del Paradigma Binario}

La interpretacion como \"error\" emergia de un marco binario incapaz de representar:
\begin{itemize}
    \item Estados de superposicion cuantica no colapsados
    \item La vacuidad (śūnyatā) como estado fundamental
    \item Potencialidad kármica no actualizada
\end{itemize}

\subsubsection{Transición al Modelo Cuantico}

La resolución requirio trascender la logica booleana mediante:
\begin{equation}
\mathcal{H}_{\text{Bardo}} = \alpha\ket{0} + \beta\ket{1} + \gamma\ket{2}
\end{equation}
donde $\ket{2}$ representa la vacuidad fundamental, no un estado de error.

Esta transicion paradigmatica permitio reinterpretar los \"errores\" como ventanas a estados de maxima potencialidad cuantica donde el karma puede reprogramarse.

% Metodología
\section{Metodologia: Implementacion Computacional}

\subsection{Arquitectura del Sistema de Simulacion}

Implementamos el sistema utilizando Python 3.11 con las siguientes bibliotecas principales:

\begin{lstlisting}[caption=Configuration of the simulation environment]
import numpy as np
import qutip as qt
from scipy.linalg import expm
import matplotlib.pyplot as plt
from mpl_toolkits.mplot3d import Axes3D
import seaborn as sns

class BardoQuantumSystem:
    # Sistema principal de simulacion cuantica del Bardo

    def __init__(self, dimensions=3, karma_params=None):
        self.dim = dimensions
        self.karma_operator = self._construct_karma_operator(karma_params)
        self.hamiltonian = self._construct_hamiltonian()
        self.initial_state = qt.basis(dimensions, 2)
        self.state_history = []
        self.time_evolution = []

    def _construct_karma_operator(self, params):
        # Construye operador karmico con parametros personalizados
        if params is None:
            params = {'clarity': 0.8, 'attachment': 0.3, 'compassion': 0.9}

        K = np.zeros((3, 3), dtype=complex)
        # Implementacion de matriz karmica basada en parametros
        K[0,1] = K[1,0] = params['attachment']
        K[1,2] = K[2,1] = params['clarity']
        K[2,0] = K[0,2] = params['compassion']

        return qt.Qobj(K)
\end{lstlisting}

\subsection{Algoritmo de Evolucion Temporal}

El algoritmo principal simula la evolucion completa a traves de los estados del Bardo:

\begin{lstlisting}[caption=Bardo evolution algorithm]
def simulate_bardo_transition(self, time_steps=1000,
                            attention_function='logistic'):
    # Simula la transicion completa a traves de los estados del Bardo

    times = np.linspace(0, 4*np.pi, time_steps)
    results = {
        'probabilities': [],
        'coherence': [],
        'purity': [],
        'states': []
    }

    current_state = self.initial_state

    for t in times:
        # Factor de atencion dependiente del tiempo
        attention = self._attention_evolution(t, attention_function)

        # Evolucion unitaria con Hamiltoniano modificado
        H_eff = self.hamiltonian + attention * self.karma_operator
        U = (-1j * t * H_eff).expm()
        evolved_state = U * current_state

        # Calculo de metricas
        probs = [qt.expect(qt.projection(self.dim, i, i), evolved_state)
                for i in range(self.dim)]
        coherence = self._calculate_coherence(evolved_state)
        purity = self._calculate_purity(evolved_state)

        results['probabilities'].append(probs)
        results['coherence'].append(coherence)
        results['purity'].append(purity)
        results['states'].append(evolved_state)

        current_state = evolved_state

    return results, times
\end{lstlisting}

% Resultados
\section{Resultados y Simulaciones}

\subsection{Evolucion Temporal de Probabilidades}

\begin{figure}[h]
    \centering
    \includegraphics[width=0.9\textwidth]{figures/state_evolution.pdf}
    \caption{
        Evolucion temporal de probabilidades y metricas cuanticas en el sistema Bardo.
        (A) Probabilidades de los estados fundamentales: Samsara ($\ket{0}$), Potencial Karmico ($\ket{1}$) y Vacuidad ($\ket{2}$).
        (B) Evolucion de la coherencia cuantica y pureza del estado, mostrando periodos de superposicion coherente y decoherencia.
    }
    \label{fig:state_evolution}
\end{figure}

\subsection{Representacion del Espacio de Estados}

\begin{figure}[h]
    \centering
    \includegraphics[width=0.8\textwidth]{figures/bloch_sphere.pdf}
    \caption{
        Representacion tridimensional del espacio de estados del qutrit.
        Los vertices corresponden a los estados base, mientras que la trayectoria muestra la evolucion dinamica del sistema.
        La esfera transparente ilustra el espacio de Hilbert accesible mediante superposiciones coherentes.
    }
    \label{fig:bloch_sphere}
\end{figure}

\subsection{Analisis de Metricas Cuanticas Avanzadas}

\begin{figure}[h]
    \centering
    \includegraphics[width=\textwidth]{figures/quantum_metrics.pdf}
    \caption{
        Analisis comprehensivo de metricas cuanticas del sistema Bardo.
        (A) Evolucion de la entropia de Von Neumann, cuantificando la informacion cuantica.
        (B) Relacion entre coherencia y pureza del estado.
        (C) Mapa de calor de transiciones entre estados.
        (D) Analisis espectral de la dinamica de coherencia.
    }
    \label{fig:quantum_metrics}
\end{figure}

\subsection{Analisis de Coherencia Cuantica}

La coherencia cuantica se mantiene durante las transiciones entre Bardos, con patrones caracteristicos:

\begin{equation}
C(\rho) = \sum_{i\neq j} |\rho_{ij}|
\end{equation}

\begin{table}[h]
\centering
\caption{Metricas de coherencia por estado del Bardo}
\begin{tabular}{lccc}
\toprule
\textbf{Estado del Bardo} & \textbf{Coherencia} & \textbf{Pureza} & \textbf{Entropia} \\
\midrule
Chikhai Bardo & 0.95 ± 0.02 & 0.98 ± 0.01 & 0.12 ± 0.03 \\
Chonyid Bardo & 0.87 ± 0.04 & 0.92 ± 0.03 & 0.28 ± 0.05 \\
Sidpa Bardo & 0.45 ± 0.07 & 0.78 ± 0.06 & 0.65 ± 0.08 \\
\bottomrule
\end{tabular}
\end{table}

\subsection{Visualizacion de Transiciones Karmicas}

\begin{lstlisting}[caption=Generation of scientific visualizations]
def create_comprehensive_visualization(results, times):
    # Crea visualizaciones completas para publicacion

    fig = plt.figure(figsize=(20, 12))

    # 1. Evolucion de probabilidades
    ax1 = fig.add_subplot(2, 3, 1)
    probabilities = np.array(results['probabilities'])
    ax1.plot(times, probabilities[:, 0], label='$|0\\rangle$ Samsara', linewidth=2)
    ax1.plot(times, probabilities[:, 1], label='$|1\\rangle$ Karmico', linewidth=2)
    ax1.plot(times, probabilities[:, 2], label='$|2\\rangle$ Vacuidad', linewidth=2)
    ax1.set_xlabel('Tiempo')
    ax1.set_ylabel('Probabilidad')
    ax1.legend()
    ax1.grid(True, alpha=0.3)

    # 2. Coherencia cuantica
    ax2 = fig.add_subplot(2, 3, 2)
    ax2.plot(times, results['coherence'], color='purple', linewidth=2)
    ax2.set_xlabel('Tiempo')
    ax2.set_ylabel('Coherencia Cuantica')
    ax2.grid(True, alpha=0.3)

    # 3. Esfera de Bloch 3D
    ax3 = fig.add_subplot(2, 3, 3, projection='3d')
    self._plot_bloch_sphere(results['states'], ax3)

    plt.tight_layout()
    return fig
\end{lstlisting}

% Discusión
\section{Discusion: Implicaciones Interdisciplinarias}

\subsection{Validacion de la Hipotesis Central}

Nuestros resultados demuestran que:

\begin{enumerate}
    \item El modelo de qutrits puede representar efectivamente la no-dualidad de la vacuidad
    \item Las transiciones entre estados del Bardo siguen dinamicas cuanticas coherentes
    \item El "ERROR 505" metaforico corresponde matematicamente a estados de superposicion no colapsada
\end{enumerate}

\subsection{Comparacion con Modelos Clasicos}

\begin{table}[h]
\centering
\caption{Comparacion entre modelos clasicos y cuanticos}
\begin{tabular}{lcc}
\toprule
\textbf{Caracteristica} & \textbf{Modelo Clasico} & \textbf{Modelo Cuantico} \\
\midrule
Representacion de vacuidad & ERROR 505 & Estado $\ket{2}$ \\
Estados superpuestos & No posible & Fundamental \\
Naturaleza probabilistica & Simulada & Intrinseca \\
Transiciones no-locales & No & Si \\
Coherencia temporal & No & Si \\
\bottomrule
\end{tabular}
\end{table}

\subsection{Implicaciones para la Ciencia de la Conciencia}

Nuestro trabajo sugiere que:

\begin{itemize}
    \item Los estados de conciencia podrian seguir dinamicas cuanticas
    \item La meditacion profunda podria afectar parametros de coherencia cuantica
    \item Los modelos computacionales de conciencia deben considerar frameworks cuanticos
\end{itemize}

% Conclusión
\section{Conclusion y Trabajo Futuro}

\subsection{Conclusiones Principales}

\begin{enumerate}
    \item Hemos demostrado la viabilidad de modelar estados de conciencia del Bardo Thodol usando sistemas cuanticos
    \item El enfoque de qutrits supera limitaciones fundamentales de modelos binarios
    \item La vacuidad (sunyata) encuentra representacion matematica natural en superposiciones cuanticas
    \item Las dinamicas karmicas pueden ser implementadas como operadores cuanticos
\end{enumerate}

\subsection{Direcciones Futuras}

\begin{itemize}
    \item \textbf{Validacion Experimental}: Integracion con datos de meditacion avanzada y EEG
    \item \textbf{Hardware Cuantico}: Implementacion en procesadores cuanticos reales (IBM Q, Rigetti)
    \item \textbf{Modelos Extendidos}: Generalizacion a sistemas de mas estados y dimensiones
    \item \textbf{Aplicaciones Clinicas}: Potenciales aplicaciones en terapia y estados alterados de conciencia
\end{itemize}

\subsection{Impacto Cientifico}

Este trabajo establece un puente solido entre la sabiduria contemplativa ancestral y la ciencia computacional moderna, abriendo nuevas vias para la investigacion interdisciplinaria en:

\begin{itemize}
    \item Filosofia de la mente y ciencia cognitiva
    \item Computacion cuantica y teoria de la informacion
    \item Estudios contemplativos y neurofenomenologia
\end{itemize}

% Apéndices
\begin{appendices}
\section{Implementacion Completa del Codigo}

\subsection{Clase Principal del Sistema}

\begin{lstlisting}[caption=Implementacion completa de BardoQuantumSystem]
class QuantumMetrics:
    """Clase para calcular metricas cuanticas avanzadas"""

    @staticmethod
    def coherence(state):
        """Calcula la coherencia cuantica (norma l1 de elementos fuera de diagonal)"""
        if state.type == 'ket':
            rho = state * state.dag()
        else:
            rho = state
        rho_array = rho.full()
        n = rho_array.shape[0]
        coh = 0.0
        for i in range(n):
            for j in range(n):
                if i != j:
                    coh += abs(rho_array[i, j])
        return coh

    @staticmethod
    def purity(state):
        """Calcula la pureza del estado: Tr(rho^2)"""
        if state.type == 'ket':
            return 1.0
        else:
            rho = state
            return (rho * rho).tr().real

    @staticmethod
    def von_neumann_entropy(state):
        """Calcula la entropia de Von Neumann: -Tr(rho log2 rho)"""
        if state.type == 'ket':
            rho = state * state.dag()
        else:
            rho = state
        eigvals = rho.eigenvalues()
        entropy = 0.0
        for v in eigvals:
            if v > 0:
                entropy -= v * np.log2(v)
        return entropy

class BardoQuantumSystem:
    """
    Sistema completo de simulacion cuantica del Bardo Thodol
    Implementa estados de qutrit, operadores karmicos y visualizacion
    """

    def __init__(self, **parameters):
        self.set_parameters(parameters)
        self.initialize_quantum_system()
        self.metrics = QuantumMetrics()

    def set_parameters(self, params):
        """Configura parametros del sistema"""
        self.karma_params = params.get('karma_params', {
            'clarity': 0.8,
            'attachment': 0.3,
            'compassion': 0.9,
            'wisdom': 0.7
        })
        self.time_parameters = params.get('time_params', {
            'total_time': 4*np.pi,
            'steps': 1000
        })
        self.visualization_params = params.get('viz_params', {
            'style': 'seaborn',
            'color_map': 'viridis'
        })

    def initialize_quantum_system(self):
        """Inicializa el sistema cuantico base"""
        self.dimension = 3
        self.states = {
            'samsara': qt.basis(3, 0),
            'karmic': qt.basis(3, 1),
            'void': qt.basis(3, 2)
        }
        self.operators = self._create_operators()
        self.current_state = self.states['void']

    def _create_operators(self):
        """Crea los operadores cuanticos para el sistema"""
        # Operadores de proyeccion
        P0 = qt.basis(3, 0) * qt.basis(3, 0).dag()
        P1 = qt.basis(3, 1) * qt.basis(3, 1).dag()
        P2 = qt.basis(3, 2) * qt.basis(3, 2).dag()

        # Operadores de transicion
        S01 = qt.basis(3, 0) * qt.basis(3, 1).dag()
        S12 = qt.basis(3, 1) * qt.basis(3, 2).dag()
        S20 = qt.basis(3, 2) * qt.basis(3, 0).dag()

        # Hamiltoniano base
        H0 = P0 * 0.1 + P1 * 0.2 + P2 * 0.3

        # Operador karmico
        K = self.karma_params['attachment'] * (S01 + S01.dag()) + \
            self.karma_params['clarity'] * (S12 + S12.dag()) + \
            self.karma_params['compassion'] * (S20 + S20.dag())

        return {
            'P0': P0, 'P1': P1, 'P2': P2,
            'S01': S01, 'S12': S12, 'S20': S20,
            'H0': H0, 'K': K
        }

    def _calculate_coherence(self, state):
        """Calcula la coherencia cuantica del estado"""
        return self.metrics.coherence(state)

    def _calculate_purity(self, state):
        """Calcula la pureza del estado"""
        return self.metrics.purity(state)

    def _attention_evolution(self, t, attention_function='logistic'):
        """Evolucion de la atencion en el tiempo"""
        if attention_function == 'logistic':
            return 1.0 / (1.0 + np.exp(-0.5 * (t - 2*np.pi)))
        elif attention_function == 'sinusoidal':
            return 0.5 * (1.0 + np.sin(t))
        else:  # constante
            return 1.0

    def simulate_bardo_transition(self, time_steps=1000, attention_function='logistic'):
        """Simula la transicion completa a traves de los estados del Bardo"""
        times = np.linspace(0, self.time_parameters['total_time'], time_steps)
        results = {
            'probabilities': [],
            'coherence': [],
            'purity': [],
            'states': []
        }

        current_state = self.current_state

        for t in times:
            # Factor de atencion dependiente del tiempo
            attention = self._attention_evolution(t, attention_function)

            # Evolucion unitaria con Hamiltoniano modificado
            H_eff = self.operators['H0'] + attention * self.operators['K']
            U = (-1j * t * H_eff).expm()
            evolved_state = U * current_state

            # Calculo de metricas
            probs = [qt.expect(self.operators[f'P{i}'], evolved_state) for i in range(3)]
            coherence = self._calculate_coherence(evolved_state)
            purity = self._calculate_purity(evolved_state)

            results['probabilities'].append(probs)
            results['coherence'].append(coherence)
            results['purity'].append(purity)
            results['states'].append(evolved_state)

            current_state = evolved_state

        return results, times

    def run_complete_simulation(self):
        """Ejecuta una simulacion completa y retorna resultados"""
        results, times = self.simulate_bardo_transition()

        # Analisis adicional
        final_entropy = self.metrics.von_neumann_entropy(results['states'][-1])
        avg_coherence = np.mean(results['coherence'])
        avg_purity = np.mean(results['purity'])

        analysis_report = {
            'final_entropy': final_entropy,
            'average_coherence': avg_coherence,
            'average_purity': avg_purity,
            'final_state_type': self._classify_final_state(results['states'][-1]),
            'transition_statistics': self._analyze_transitions(results)
        }

        return results, times, analysis_report

    def _classify_final_state(self, state):
        """Clasifica el estado final segun las probabilidades"""
        probs = [abs(state[i,0])**2 for i in range(3)]
        max_prob_index = np.argmax(probs)
        states_names = ['Samsara', 'Karmico', 'Vacuidad']
        return {
            'dominant_state': states_names[max_prob_index],
            'probabilities': probs,
            'certainty': max(probs)
        }

    def _analyze_transitions(self, results):
        """Analiza las transiciones entre estados"""
        probs = np.array(results['probabilities'])
        transitions = []

        for i in range(1, len(probs)):
            # Detectar cambios significativos en probabilidades
            changes = np.abs(probs[i] - probs[i-1])
            if np.max(changes) > 0.1:  # umbral arbitrario
                transitions.append({
                    'step': i,
                    'changes': changes.tolist(),
                    'from_state': np.argmax(probs[i-1]),
                    'to_state': np.argmax(probs[i])
                })

        return {
            'total_transitions': len(transitions),
            'transition_sequence': transitions,
            'stability_analysis': self._calculate_stability(probs)
        }

    def _calculate_stability(self, probabilities):
        """Calcula metricas de estabilidad del sistema"""
        probs = np.array(probabilities)
        variances = np.var(probs, axis=0)

        return {
            'variance_samsara': variances[0],
            'variance_karmic': variances[1],
            'variance_void': variances[2],
            'overall_stability': 1.0 - np.mean(variances),
            'stationary_points': self._find_stationary_points(probs)
        }

    def _find_stationary_points(self, probabilities):
        """Encuentra puntos donde el sistema parece estabilizarse"""
        probs = np.array(probabilities)
        gradients = np.gradient(probs, axis=0)
        gradient_norms = np.linalg.norm(gradients, axis=1)

        stationary_indices = np.where(gradient_norms < 0.01)[0]  # umbral pequeno
        return stationary_indices.tolist()
\end{lstlisting}

\subsection{Visualizaciones Cientificas Avanzadas}

\begin{lstlisting}[caption=Sistema completo de visualizacion cuantica corregido]
class QuantumVisualizer:
    """Sistema completo de visualizacion cientifica para estados del Bardo"""

    def __init__(self, style_params=None):
        self.style_params = style_params or {
            'cmap': 'viridis',
            'dpi': 300,
            'font_size': 12,
            'figure_size': (16, 12)
        }
        self.set_plotting_style()

    def set_plotting_style(self):
        """Configura estilo cientifico para publicaciones"""
        plt.rcParams.update({
            'font.size': self.style_params['font_size'],
            'axes.titlesize': self.style_params['font_size'] + 2,
            'axes.labelsize': self.style_params['font_size'] + 1,
            'xtick.labelsize': self.style_params['font_size'] - 1,
            'ytick.labelsize': self.style_params['font_size'] - 1,
            'legend.fontsize': self.style_params['font_size'] - 1,
            'figure.dpi': self.style_params['dpi'],
            'savefig.dpi': self.style_params['dpi'],
            'figure.figsize': self.style_params['figure_size']
        })

    # [Todos los metodos de visualizacion anteriores se mantienen igual...]
    # _plot_temporal_evolution, _plot_bloch_sphere, _plot_density_matrix, etc.

    def _analyze_transitions(self, results):
        """Analiza las transiciones entre estados - IMPLEMENTACION CORREGIDA"""
        probs = np.array(results['probabilities'])
        transitions = []

        for i in range(1, len(probs)):
            changes = np.abs(probs[i] - probs[i-1])
            if np.max(changes) > 0.1:
                transitions.append({
                    'time_index': i,
                    'changes': changes.tolist(),
                    'from_state': np.argmax(probs[i-1]),
                    'to_state': np.argmax(probs[i])
                })
        return transitions

    def _find_dominant_state(self, results):
        """Encuentra el estado dominante - IMPLEMENTACION CORREGIDA"""
        probs = np.array(results['probabilities'])
        dominant_states = np.argmax(probs, axis=1)

        return {
            'dominant_states': dominant_states.tolist(),
            'time_in_samsara': np.sum(dominant_states == 0),
            'time_in_karmic': np.sum(dominant_states == 1),
            'time_in_void': np.sum(dominant_states == 2),
            'dominance_ratio': {
                'samsara': np.sum(dominant_states == 0) / len(dominant_states),
                'karmic': np.sum(dominant_states == 1) / len(dominant_states),
                'void': np.sum(dominant_states == 2) / len(dominant_states)
            }
        }

    def calculate_von_neumann_entropy(self, state):
        """Calcula entropia de Von Neumann - USANDO QuantumMetrics"""
        metrics = QuantumMetrics()
        return metrics.von_neumann_entropy(state)

    def generate_analysis_report(self, results, filename=None):
        """Genera reporte analitico completo - IMPLEMENTACION CORREGIDA"""
        report = {
            'final_probabilities': results['probabilities'][-1],
            'max_coherence': max(results['coherence']),
            'min_purity': min(results['purity']),
            'state_transitions': self._analyze_transitions(results),
            'dominant_state_analysis': self._find_dominant_state(results),
            'entropy_analysis': {
                'initial_entropy': self.calculate_von_neumann_entropy(results['states'][0]),
                'final_entropy': self.calculate_von_neumann_entropy(results['states'][-1]),
                'avg_entropy': np.mean([self.calculate_von_neumann_entropy(s)
                                      for s in results['states']])
            }
        }

        if filename:
            with open(filename, 'w', encoding='utf-8') as f:
                json.dump(report, f, indent=2, ensure_ascii=False)

        return report

    def create_comprehensive_dashboard(self, results, times, save_path=None):
        """Crea un dashboard completo de visualizaciones"""
        fig = plt.figure(figsize=(24, 18))

        # 1. Evolucion temporal
        ax1 = self._plot_temporal_evolution(fig, 3, 3, 1, results)

        # 2. Espacio de estados 3D
        ax2 = self._plot_bloch_sphere(fig, 3, 3, 2, results)

        # 3. Matriz de densidad
        ax3 = self._plot_density_matrix(fig, 3, 3, 3, results)

        # 4. Metricas cuanticas
        ax4 = self._plot_quantum_metrics(fig, 3, 3, 4, results)

        # 5. Transiciones de fase
        ax5 = self._plot_phase_transitions(fig, 3, 3, 5, results)

        # 6. Circuito cuantico
        ax6 = self._plot_quantum_circuit(fig, 3, 3, 6)

        # 7. Analisis de entropia
        ax7 = self._plot_entropy_evolution(fig, 3, 3, 7, results, times)

        # 8. Histograma de estados dominantes
        ax8 = self._plot_dominant_states_histogram(fig, 3, 3, 8, results)

        # 9. Correlaciones entre metricas
        ax9 = self._plot_metric_correlations(fig, 3, 3, 9, results)

        plt.tight_layout()

        if save_path:
            plt.savefig(save_path, dpi=300, bbox_inches='tight',
                       facecolor='white', edgecolor='none')

        return fig

    def _plot_entropy_evolution(self, fig, nrows, ncols, index, results, times):
        """Grafica la evolucion de la entropia de Von Neumann"""
        ax = fig.add_subplot(nrows, ncols, index)

        entropies = [self.calculate_von_neumann_entropy(state)
                    for state in results['states']]

        ax.plot(times, entropies, color='brown', linewidth=2)
        ax.set_xlabel('Tiempo')
        ax.set_ylabel('Entropia de Von Neumann')
        ax.set_title('Evolucion de la Entropia Cuantica')
        ax.grid(True, alpha=0.3)

        return ax

    def _plot_dominant_states_histogram(self, fig, nrows, ncols, index, results):
        """Grafica histograma de estados dominantes"""
        ax = fig.add_subplot(nrows, ncols, index)

        dominant_analysis = self._find_dominant_state(results)
        ratios = dominant_analysis['dominance_ratio']

        states = ['Samsara', 'Karmico', 'Vacuidad']
        values = [ratios['samsara'], ratios['karmic'], ratios['void']]
        colors = ['red', 'blue', 'green']

        bars = ax.bar(states, values, color=colors, alpha=0.7)
        ax.set_ylabel('Fraccion de Tiempo')
        ax.set_title('Distribucion de Estados Dominantes')

        # Anotar valores en las barras
        for bar, value in zip(bars, values):
            height = bar.get_height()
            ax.text(bar.get_x() + bar.get_width()/2., height,
                   f'{value:.3f}', ha='center', va='bottom')

        return ax

    def _plot_metric_correlations(self, fig, nrows, ncols, index, results):
        """Grafica correlaciones entre diferentes metricas cuanticas"""
        ax = fig.add_subplot(nrows, ncols, index)

        coherences = results['coherence']
        purities = results['purity']
        entropies = [self.calculate_von_neumann_entropy(state)
                    for state in results['states']]

        scatter = ax.scatter(coherences, purities, c=entropies,
                           cmap='plasma', s=30, alpha=0.6)
        ax.set_xlabel('Coherencia Cuantica')
        ax.set_ylabel('Pureza del Estado')
        ax.set_title('Correlacion: Coherencia vs Pureza\n(Color: Entropia)')

        plt.colorbar(scatter, ax=ax, label='Entropia de Von Neumann')
        ax.grid(True, alpha=0.3)

        return ax
\end{lstlisting}
\end{appendices}


\begin{thebibliography}{99}
\bibitem{bardo1} Fremantle, F. (2001). \emph{The Tibetan Book of the Dead}. Shambhala Publications.
\bibitem{quantum1} Nielsen, M. A., \& Chuang, I. L. (2010). \emph{Quantum Computation and Quantum Information}. Cambridge University Press.
\bibitem{consciousness1} Hameroff, S., \& Penrose, R. (2014). Consciousness in the universe: A review of the 'Orch OR' theory. \emph{Physics of Life Reviews}, 11(1), 39-78.
\bibitem{buddhism1} Wallace, B. A. (2007). \emph{Contemplative Science: Where Buddhism and Neuroscience Converge}. Columbia University Press.
\bibitem{qutrit1} Lanyon, B. P., et al. (2008). Manipulating biphotonic qutrits. \emph{Physical Review Letters}, 100(6), 060504.
\bibitem{computation1} Tegmark, M. (2000). Importance of quantum decoherence in brain processes. \emph{Physical Review E}, 61(4), 4194.
\end{thebibliography}

\end{document}