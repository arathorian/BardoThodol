\documentclass[12pt,a4paper]{article}
\usepackage[utf8]{inputenc}
\usepackage[spanish]{babel}
\usepackage{amsmath, amssymb, amsthm}
\usepackage{quantikz}
\usepackage{graphicx}
\usepackage{subcaption}
\usepackage[colorlinks=true]{hyperref}
\usepackage{pgfplots}
\pgfplotsset{compat=1.18}
\usepackage{tikz}
\usetikzlibrary{quantikz, arrows.meta, positioning}
\usepackage{pgf}
\usepackage{siunitx}

% Comandos para autogeneración de gráficos
\newcommand{\generategraphs}{%
    \immediate\write18{python3 scripts/generate_quantum_plots.py}%
    \immediate\write18{python3 scripts/generate_state_evolution.py}%
    \immediate\write18{python3 scripts/generate_error505_analysis.py}%
}

\title{Bardo Thödol Quantum Framework: Modelado Computacional Cuántico de Estados Post-Mortem mediante Qutrits}
\author{Horacio Héctor Hamann\\\href{https://github.com/arathorian/BardoThodol}{https://github.com/arathorian/BardoThodol}}
\date{Julio 2025}

\begin{document}

% Ejecutar generación de gráficos al inicio
\generategraphs

\maketitle

\begin{abstract}
[Abstract mantiene igual...]
\end{abstract}

\section{Introducción: Marco Conceptual Completo}
[Contenido introductorio igual...]

\section{Marco Teórico Expandido}

\subsection{Sistema de Qutrits Completo con Visualización Automática}

\begin{equation}
|\psi\rangle = \alpha|0\rangle + \beta|1\rangle + \gamma|2\rangle
\end{equation}

\begin{figure}[htbp]
\centering
\begin{tikzpicture}
    \begin{axis}[
        width=0.8\textwidth,
        height=0.6\textwidth,
        view={45}{30},
        xlabel={$|0\rangle$},
        ylabel={$|1\rangle$},
        zlabel={$|2\rangle$},
        grid=major]
        \addplot3[surf, opacity=0.6] file {data/qutrit_surface.dat};
        \addplot3[quiver={u=\thisrow{u}, v=\thisrow{v}, w=\thisrow{w}}] file {data/quantum_vectors.dat};
    \end{axis}
\end{tikzpicture}
\caption{Espacio de Hilbert tri-dimensional del qutrit generado automáticamente}
\label{fig:qutrit_space}
\end{figure}

\subsection{Operadores Kármicos con Evolución Temporal Visualizada}

\begin{equation}
\hat{K}(\kappa, \alpha) = e^{-i(\kappa \hat{H}_k + \alpha \hat{H}_a)t}
\end{equation}

\begin{figure}[htbp]
\centering
\begin{tikzpicture}
    \begin{axis}[
        width=\textwidth,
        height=0.4\textwidth,
        xlabel={Tiempo (iteraciones)},
        ylabel={Probabilidad},
        legend pos=north east,
        grid=major]
        \addplot table[x=time,y=prob0] {data/state_evolution.dat};
        \addplot table[x=time,y=prob1] {data/state_evolution.dat};
        \addplot table[x=time,y=prob2] {data/state_evolution.dat};
        \addplot[red, thick] table[x=time,y=attention] {data/state_evolution.dat};
        \legend{$|0\rangle$,$|1\rangle$,$|2\rangle$,$Atención$}
    \end{axis}
\end{tikzpicture}
\caption{Evolución temporal de probabilidades de estados bajo operadores kármicos}
\label{fig:state_evolution}
\end{figure}

\section{Análisis del ERROR 505 como Estado de Vacuidad}

\begin{figure}[htbp]
\centering
\begin{tikzpicture}
    \begin{axis}[
        width=0.9\textwidth,
        height=0.5\textwidth,
        xlabel={Carga Kármica ($\kappa$)},
        ylabel={Atención ($\alpha$)},
        zlabel={Probabilidad $|2\rangle$},
        colorbar,
        grid=major]
        \addplot3[surf] file {data/error505_surface.dat};
    \end{axis}
\end{tikzpicture}
\caption{Superficie de probabilidad del estado $|2\rangle$ (ERROR 505) en función de parámetros kármicos}
\label{fig:error505_surface}
\end{figure}

\begin{figure}[htbp]
\centering
\begin{tikzpicture}
    \begin{axis}[
        width=0.8\textwidth,
        height=0.5\textwidth,
        xlabel={Coherencia Cuántica},
        ylabel={Entropía von Neumann},
        zlabel={Estado $|2\rangle$ Dominante},
        only marks,
        point meta=\thisrow{state2}]
        \addplot3[scatter] file {data/quantum_metrics.dat};
    \end{axis}
\end{tikzpicture}
\caption{Relación entre coherencia cuántica, entropía y dominancia del estado de vacuidad}
\label{fig:quantum_metrics}
\end{figure}

\section{Implementación Técnica con Scripts de Generación}

\subsection{Sistema de Generación Automática de Gráficos}

\begin{verbatim}
BardoThodol/
├── scripts/                    # SCRIPTS DE GENERACIÓN (NUEVO)
│   ├── generate_quantum_plots.py
│   ├── generate_state_evolution.py
│   ├── generate_error505_analysis.py
│   └── latex_graph_utils.py
├── data/                       # DATOS GENERADOS (NUEVO)
│   ├── qutrit_surface.dat
│   ├── state_evolution.dat
│   ├── error505_surface.dat
│   └── quantum_metrics.dat
└── figures/                    # FIGURAS EXPORTADAS (NUEVO)
    ├── quantum_circuits.pdf
    ├── state_transitions.pdf
    └── statistical_analysis.pdf
\end{verbatim}

\subsection{Script Principal de Generación de Gráficos}

\begin{verbatim}
# scripts/generate_quantum_plots.py
import numpy as np
import matplotlib.pyplot as plt
from core.quantum_state import QuantumState
from simulation.advanced_simulator import AdvancedBardoSimulator

def generate_qutrit_surface_data():
    """Genera datos para superficie de espacio de Hilbert"""
    alpha_vals = np.linspace(0, 1, 50)
    beta_vals = np.linspace(0, 1, 50)

    with open('data/qutrit_surface.dat', 'w') as f:
        for alpha in alpha_vals:
            for beta in beta_vals:
                gamma = np.sqrt(1 - alpha**2 - beta**2)
                if gamma >= 0:  # Solo puntos válidos
                    f.write(f"{alpha} {beta} {gamma}\n")
                f.write("\n")  # Separador para pgfplots

def generate_state_evolution_data():
    """Genera datos de evolución temporal"""
    simulator = AdvancedBardoSimulator()
    results = simulator.run_comprehensive_simulation(samples=100)

    with open('data/state_evolution.dat', 'w') as f:
        f.write("time prob0 prob1 prob2 attention\n")
        for i, result in enumerate(results):
            f.write(f"{i} {result['prob0']} {result['prob1']} {result['prob2']} {result['attention']}\n")

def generate_error505_analysis():
    """Análisis detallado del ERROR 505"""
    kappa_range = np.linspace(0, 1, 20)
    attention_range = np.linspace(0, 1, 20)

    with open('data/error505_surface.dat', 'w') as f:
        for kappa in kappa_range:
            for attention in attention_range:
                state = QuantumState(0.5, 0.3, 0.2)
                state.apply_karma_operator(kappa, attention)
                prob2 = abs(state.gamma)**2
                f.write(f"{kappa} {attention} {prob2}\n")
            f.write("\n")

if __name__ == "__main__":
    generate_qutrit_surface_data()
    generate_state_evolution_data()
    generate_error505_analysis()
\end{verbatim}

\section{Circuitos Cuánticos con Quantikz}

\begin{figure}[htbp]
\centering
\begin{quantikz}
    \lstick{$\ket{0}$} & \gate{H} & \ctrl{1} & \gate{R_y(\theta)} & \meter{} & \cw \cwbend{1} \\
    \lstick{$\ket{0}$} & \gate{H} & \targ{} & \gate{R_z(\phi)} & \meter{} & \cw \\
    \lstick{$\ket{0}$} & \gate{X} & \ctrl{-1} & \gate{R_x(\psi)} & \meter{} & \cw
\end{quantikz}
\caption{Circuito cuántico para simulación de transiciones kármicas entre estados de qutrit}
\label{fig:quantum_circuit}
\end{figure}

\section{Análisis Estadístico con Gráficos Automáticos}

\begin{figure}[htbp]
\centering
\begin{tikzpicture}
    \begin{axis}[
        width=0.9\textwidth,
        height=0.6\textwidth,
        xlabel={Muestras},
        ylabel={Distribución de Probabilidad},
        ymin=0, ymax=1,
        grid=major,
        legend style={at={(0.5,-0.2)}, anchor=north}]

        \addplot+[hist={data=x, bins=20, data min=0, data max=1}]
            file {data/probability_distribution0.dat};
        \addplot+[hist={data=x, bins=20, data min=0, data max=1}]
            file {data/probability_distribution1.dat};
        \addplot+[hist={data=x, bins=20, data min=0, data max=1}]
            file {data/probability_distribution2.dat};

        \legend{Estado $|0\rangle$, Estado $|1\rangle$, Estado $|2\rangle$}
    \end{axis}
\end{tikzpicture}
\caption{Distribución estadística de probabilidades de estados en 1000 simulaciones}
\label{fig:probability_distribution}
\end{figure}

\section{Configuración de Compilación Automática}

\subsection{Makefile para Generación Automática}

\begin{verbatim}
# Makefile
.PHONY: all graphs paper clean

all: graphs paper

graphs:
    python3 scripts/generate_quantum_plots.py
    python3 scripts/generate_state_evolution.py
    python3 scripts/generate_error505_analysis.py

paper: graphs
    pdflatex -shell-escape main.tex
    pdflatex -shell-escape main.tex  # Para referencias cruzadas

clean:
    rm -f data/*.dat
    rm -f figures/*.pdf
    rm -f main.aux main.log main.out
\end{verbatim}

\subsection{Configuración de Dependencias}

\begin{verbatim}
# requirements_latex.txt
numpy>=1.21.0
matplotlib>=3.5.0
qiskit>=0.34.0
scipy>=1.7.0
pandas>=1.3.0
\end{verbatim}

\section{Resultados y Visualización Científica}

Los gráficos generados automáticamente proporcionan:
\begin{itemize}
    \item \textbf{Validación visual inmediata} de los resultados teóricos
    \item \textbf{Actualización en tiempo real} al modificar parámetros
    \item \textbf{Calidad de publicación} con gráficos vectoriales nativos
    \item \textbf{Reproducibilidad completa} del análisis científico
\end{itemize}

\begin{figure}[htbp]
\centering
\begin{tikzpicture}
    \begin{axis}[
        width=0.8\textwidth,
        height=0.5\textwidth,
        xlabel={Parámetro de Atención ($\alpha$)},
        ylabel={Tasa de Transición $|0\rangle \rightarrow |2\rangle$},
        error bars/y dir=both,
        error bars/y explicit]
        \addplot table[x=attention,y=transition_rate,y error=error]
            {data/attention_transition.dat};
    \end{axis}
\end{tikzpicture}
\caption{Influencia del parámetro de atención en las transiciones hacia vacuidad}
\label{fig:attention_transition}
\end{figure}

\end{document}