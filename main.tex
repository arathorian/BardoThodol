\documentclass[12pt, a4paper]{article}

\usepackage[spanish,es-noshorthands]{babel}
\usepackage[utf8]{inputenc}
\usepackage[T1]{fontenc}
\usepackage{amsmath, amssymb, amsthm, mathrsfs}
\usepackage{physics}
\usepackage{microtype}
\usepackage{siunitx}
\sisetup{
    output-decimal-marker = {,},
    group-separator = {.},
    per-mode = symbol
}

\usepackage{graphicx, wrapfig, booktabs, multirow, multicol, xcolor}
    \graphicspath{{./}{./figures/}}
\usepackage{geometry, abstract, caption, subcaption}
\usepackage{listings}
\usepackage[hyphens]{url}
\usepackage{hyperref}
\usepackage{appendix}
\hypersetup{
    colorlinks=true,
    linkcolor=blue,
    filecolor=magenta,
    urlcolor=cyan,
    pdftitle={Simulación Cuántica del Bardo Thodol},
    pdfauthor={Horacio Hector Hamann},
    pdfsubject={Computación Cuántica y Filosofía Budista},
    pdfkeywords={Bardo Thodol, Computación Cuántica, Qutrits, Conciencia, Simulación},
    breaklinks=true,
    pdfencoding=auto
}
\usepackage{fancyhdr}

% Quantikz después de physics
\usepackage{tikz}
\usetikzlibrary{quantikz}

% Configuración de geometría
\geometry{left=2.5cm, right=2.5cm, top=2.5cm, bottom=2.5cm}

% Solución completa: fancyhdr + contador de páginas
\setlength{\headheight}{14.5pt}

% Colores personalizados
\definecolor{deepblue}{rgb}{0,0.2,0.6}
\definecolor{deepred}{rgb}{0.6,0,0}
\definecolor{deepgreen}{rgb}{0,0.5,0}
\definecolor{codegray}{rgb}{0.5,0.5,0.5}
\definecolor{paradoxcolor}{rgb}{0.8,0.4,0}

% Configuración de Listings

\lstset{
    language=Python,
    basicstyle=\ttfamily\footnotesize,
    keywordstyle=\color{deepblue}\bfseries,
    commentstyle=\color{deepgreen}\itshape,
    stringstyle=\color{deepred},
    numbers=left,
    numberstyle=\tiny\color{codegray},
    stepnumber=1,
    numbersep=8pt,
    backgroundcolor=\color{white},
    showspaces=false,
    showstringspaces=false,
    showtabs=false,
    tabsize=4,
    captionpos=b,
    breaklines=true,
    breakatwhitespace=true,
    breakindent=20pt,
    frame=single,
    rulecolor=\color{codegray},
    xleftmargin=17pt,
    framexleftmargin=17pt,
    escapeinside={(*@}{@*)},
    inputencoding=utf8,
    extendedchars=true,
    literate=
        {á}{{\'a}}1 {é}{{\'e}}1 {í}{{\'i}}1 {ó}{{\'o}}1 {ú}{{\'u}}1
        {Á}{{\'A}}1 {É}{{\'E}}1 {Í}{{\'I}}1 {Ó}{{\'O}}1 {Ú}{{\'U}}1
        {ñ}{{\~n}}1 {Ñ}{{\~N}}1
        {ü}{{\"u}}1 {Ü}{{\"U}}1
}

% Configuración de encabezados
\pagestyle{fancy}
\fancyhf{}
\fancyhead[L]{\small Simulacion Cuantica del Bardo Thodol}
\fancyhead[R]{\small \thepage}
\renewcommand{\headrulewidth}{0.4pt}

% COMANDOS:
\newcommand{\expected}[1]{\langle #1 \rangle}
\newcommand{\state}[1]{\ket{\psi_{#1}}}
\newcommand{\hamiltonian}{\hat{H}}
\newcommand{\unitary}{\hat{U}}
\newcommand{\karmaop}{\hat{K}}
\newcommand{\vacuumstate}{\ket{2}}
\newcommand{\manifeststate}{\ket{0}}
\newcommand{\potentialstate}{\ket{1}}

% Teoremas y definiciones
\newtheorem{teorema}{Teorema}
\newtheorem{definicion}{Definicion}
\newtheorem{corolario}{Corolario}
\newtheorem{lema}{Lema}
\newtheorem{paradoja}{Paradoja}[section]

% Nuevo entorno para paradojas epistemológicas
\newenvironment{epistemic_warning}[1]{
    \begin{center}
    \begin{minipage}{0.9\textwidth}
    \color{paradoxcolor}
    \hrule
    \vspace{0.2cm}
    \textbf{ADVERTENCIA EPISTEMOLÓGICA: #1}
    \vspace{0.1cm}
    \hrule
    \vspace{0.2cm}
}{
    \vspace{0.2cm}
    \hrule
    \end{minipage}
    \end{center}
}

\title{
    \textbf{Simulacion Cuantica de los Estados de Conciencia del Bardo Thodol:} \\
    \large Un Enfoque Computacional desde la Teoria de Qutrits y Dinamica Karmica \\
    \normalsize Con Transparencia Epistemológica sobre Límites del Modelado
}

\author{
    \textbf{Horacio Hector Hamann} \\
    \small \href{https://github.com/arathorian/BardoThodol}{https://github.com/arathorian/BardoThodol}
}

\date{Noviembre 2025}

\begin{document}

% Portada
\begin{titlepage}
    \centering
    \vspace*{1cm}
    
    {\Huge \textbf{Simulacion Cuantica del Bardo Thodol}}
    
    \vspace{0.5cm}
    {\Large Modelado de Estados de Conciencia Post-Mortem \\ mediante Sistemas de Qutrits y Operadores Karmicos}
    
    \vspace{0.3cm}
    {\normalsize \textit{Con Metamodelado Epistemológico Explícito}}
    
    \vspace{1.5cm}
    
    \begin{figure}[h]
        \centering
        \begin{quantikz}
            \lstick{$\ket{0}$} & \gate{H} & \ctrl{1} & \gate{R_y(\theta)} & \meter{} \\
            \lstick{$\ket{2}$} & \gate{S} & \targ{} & \gate{R_z(\phi)} & \meter{} \\
            \lstick{$\ket{1}$} & \gate{T} & \ctrl{-1} & \gate{H} & \meter{}
        \end{quantikz}
        \caption{Circuito cuantico representando transiciones entre estados del Bardo}
    \end{figure}

    \vspace{1.5cm}

    {\large \textbf{Horacio Hector Hamann}}

    \vspace{0.3cm}
    {\small \href{https://github.com/arathorian/BardoThodol}{https://github.com/arathorian/BardoThodol}}

    \vspace{1cm}

    \begin{abstract}
        \noindent Este articulo presenta un marco teorico y computacional innovador para la simulacion cuantica de los estados de conciencia descritos en el \emph{Bardo Thodol} (Libro Tibetano de los Muertos). Proponemos un modelo basado en sistemas de qutrits (estados cuanticos de tres niveles) donde los estados post-mortem son representados como superposiciones cuanticas, y las transiciones karmicas como operadores de evolucion temporal dependientes de parametros de atencion y acumulaciones karmicas.

        \vspace{0.3cm}
        \noindent Siguiendo el método Madhyamaka de las Dos Verdades, este trabajo explicita las paradojas irresolibles inherentes al modelado matemático de fenómenos contemplativos, distinguiendo entre verdad convencional (\emph{saṃvṛti-satya}), verdad última (\emph{paramārtha-satya}) y uso pedagógico (\emph{upāya}).

        \vspace{0.3cm}
        \noindent \textbf{Palabras clave:} Bardo Thodol, Computacion Cuantica, Qutrits, Estados de Conciencia, Simulacion, Sunyata, Karma, Decoherencia Cuantica, Epistemología del Modelado
    \end{abstract}

    \vfill
    {\small Proyecto iniciado en Enero 2025 · Actualizado en Noviembre 2025}
\end{titlepage}

\cleardoublepage
\pagenumbering{roman}
\setcounter{page}{1}

% Índice
\tableofcontents

\cleardoublepage
\pagenumbering{arabic}
\setcounter{page}{1}

% Introducción
\section{Introduccion: Del Texto Sagrado al Algoritmo Cuantico}

\subsection{Contexto Interdisciplinario}

El \emph{Bardo Thodol}, tradicionalmente interpretado como una guia ritual para la transicion post-mortem en la tradicion tibetana, es reformulado en este trabajo como un \textbf{algoritmo ancestral} que codifica la dinamica fundamental de estados de conciencia. Esta reinterpretacion se situa en la interseccion de:

\begin{itemize}
    \item \textbf{Filosofia Budista Mahayana}: Especialmente la doctrina de la vacuidad (sunyata) y la naturaleza budica
    \item \textbf{Computacion Cuantica}: Sistemas de multiples estados y dinamicas de coherencia-decoherencia
    \item \textbf{Neurofenomenologia}: Estudio cientifico de los estados de conciencia
    \item \textbf{Teoria de la Informacion}: Procesamiento y transicion de estados informacionales
    \item \textbf{Epistemología Crítica}: Análisis reflexivo de límites del modelado formal
\end{itemize}

\subsection{Hipotesis Central}

Formulamos nuestra hipotesis fundamental como:

\begin{definicion}[Hipotesis de Simulacion Cuantica del Bardo]
El Bardo Thodol puede ser modelado como un sistema cuantico de multiples estados donde:
\begin{equation}
\mathcal{H}_{\text{Bardo}} = \alpha\ket{0} + \beta\ket{1} + \gamma\ket{2}
\end{equation}
con $\ket{0}$ representando el estado de realidad manifiesta (samsara), $\ket{1}$ estados potenciales karmicos, y $\ket{2}$ la vacuidad fundamental (sunyata), donde $|\alpha|^2 + |\beta|^2 + |\gamma|^2 = 1$.
\end{definicion}

\begin{epistemic_warning}{Nivel de Verdad}
Esta representación matemática pertenece al nivel convencional (\emph{saṃvṛti-satya}). En el nivel último (\emph{paramārtha-satya}), śūnyatā no es un estado vectorial mesurable sino la naturaleza vacía de todos los fenómenos, incluido el propio concepto de vacuidad. El modelo es \textbf{upāya} (medio hábil) pedagógico, no descripción ontológica.
\end{epistemic_warning}

\subsection{Justificacion Cientifica}

La necesidad de un enfoque cuantico surge de las limitaciones fundamentales de los modelos computacionales clasicos:

\begin{itemize}
    \item \textbf{Problema del Dualismo}: Los sistemas binarios no pueden capturar la naturaleza no-dual de la vacuidad
    \item \textbf{Limitaciones de Turing}: La maquina clasica no puede representar superposiciones coherentes
    \item \textbf{Naturaleza Probabilistica}: El proceso karmico es intrinsecamente probabilistico, no determinista
\end{itemize}

\subsection{Transparencia Epistemologica: Limites del Modelado}

Este trabajo adopta el método de las \textbf{Dos Verdades} (Madhyamaka) aplicado a la simulación computacional:

\begin{definicion}[Niveles de Verdad en el Modelo]
\begin{enumerate}
    \item \textbf{Nivel Convencional} (\emph{saṃvṛti-satya}): Las matemáticas cuánticas son válidas formalmente en su dominio
    \item \textbf{Nivel Último} (\emph{paramārtha-satya}): El formalismo no captura śūnyatā como realidad última
    \item \textbf{Uso Pedagógico} (\emph{upāya}): El modelo es herramienta heurística para exploración, no identidad con el fenómeno
\end{enumerate}
\end{definicion}

\subsubsection{Paradojas Irresolibles Documentadas}

\begin{paradoja}[Cuantificación Kármica]
\label{paradox:karma}
Asignar valores numéricos al karma (e.g., $k_{\text{clarity}}=0.8$, $k_{\text{attachment}}=0.3$) \textbf{reifica} lo que el Abhidharma describe como flujo impermanente (\emph{anitya}) sin sustancia fija (\emph{anātman}).

\textbf{Brecha irreducible:} El karma en Madhyamaka carece de \emph{svabhāva} (naturaleza inherente), siendo proceso de originación interdependiente (\emph{pratītyasamutpāda}), no magnitud medible.

\textbf{Valor pedagógico:} Los parámetros permiten explorar cómo diferentes tendencias habituales afectan transiciones, sin afirmar que el karma \emph{es} estos números. Función heurística, no descriptiva.
\end{paradoja}

\begin{paradoja}[Reificación de Vacuidad]
\label{paradox:sunyata}
Representar śūnyatā como vector $\ket{2} = [0,0,1]^T$ en espacio de Hilbert contradice su naturaleza de \emph{niḥsvabhāva} (ausencia de ser inherente).

\textbf{Brecha irreducible:} Convertir vacuidad en estado matemático separado es exactamente el tipo de cosificación (\emph{saṃjñā}) que el Prajñāpāramitā advierte evitar. Es contradicción performativa irresoluble.

\textbf{Valor pedagógico:} Demuestra la necesidad de frameworks no-binarios que superen lógica clásica. $\ket{2}$ no \emph{es} vacuidad, \emph{señala} hacia ella como dedo apuntando a la luna.
\end{paradoja}

\begin{paradoja}[Temporalidad del Modelo]
\label{paradox:time}
La evolución temporal $\unitary(t) = e^{-i\hamiltonian t}$ requiere tiempo como parámetro continuo, mientras que en estados meditativos profundos (\emph{samādhi}), la experiencia temporal colapsa.

\textbf{Brecha irreducible:} El formalismo matemático no puede modelar experiencia atemporal sin contradecirse estructuralmente. \emph{Kāla} (tiempo) es construcción mental, no absoluto.

\textbf{Valor pedagógico:} Muestra dinámica de transiciones como proceso secuencial útil para comprensión conceptual. El usuario debe recordar que el tiempo matemático es artefacto del modelo.
\end{paradoja}

\begin{table}[h]
\centering
\caption{Metamodelado: Verdad Convencional vs Verdad Última}
\label{tab:two_truths}
\begin{tabular}{p{3cm}p{5.5cm}p{5.5cm}}
\toprule
\textbf{Aspecto} & \textbf{Saṃvṛti (Convencional)} & \textbf{Paramārtha (Último)} \\
\midrule
Vacuidad & Vector $\ket{2} = [0,0,1]^T$ & \emph{Niḥsvabhāva} sin sustancia \\
Karma & Operador $\karmaop$ con parámetros numéricos & \emph{Pratītyasamutpāda} no-cuantificable \\
Tiempo & Parámetro $t \in \mathbb{R}$ & Construcción mental (\emph{kāla}) \\
Medición & Colapso $\ket{\psi} \to \ket{i}$ & \emph{Rigpa} no-dual sin observador \\
Utilidad & Válida formalmente & Herramienta (\emph{upāya}) \\
\bottomrule
\end{tabular}
\end{table}

% Marco Teórico
\section{Marco Teorico: Fundamentos Cuanticos y Filosoficos}

\subsection{Sistema de Qutrits para Estados de Conciencia}

Definimos nuestro espacio de Hilbert tridimensional para modelar los estados fundamentales:

\begin{equation}
\mathcal{H} = \text{span}\{\ket{0}, \ket{1}, \ket{2}\}
\end{equation}

Con los operadores de proyeccion correspondientes:

\begin{equation}
P_i = \ket{i}\bra{i}, \quad i \in \{0,1,2\}
\end{equation}

\begin{definicion}[Estados Fundamentales - Nivel Convencional]
\begin{align*}
\ket{0} &= \begin{bmatrix} 1 \\ 0 \\ 0 \end{bmatrix} \quad \text{(Realidad manifiesta - Samsara)} \\
\ket{1} &= \begin{bmatrix} 0 \\ 1 \\ 0 \end{bmatrix} \quad \text{(Potencial karmico - Estados latentes)} \\
\ket{2} &= \begin{bmatrix} 0 \\ 0 \\ 1 \end{bmatrix} \quad \text{(Señala hacia Sunyata)}
\end{align*}
\end{definicion}

\begin{epistemic_warning}{Interpretación de Estados Base}
Los tres vectores base NO son realidades ontológicamente separadas. En el nivel último, todos los estados interpenetran sin frontera fija. La separación matemática es convención pedagógica para análisis formal.
\end{epistemic_warning}

\subsection{Hamiltoniano Karmico y Operadores de Evolucion}

El operador de evolucion incorpora parametros karmicos y de atencion:

\begin{equation}
\hamiltonian_K = \sum_{i\neq j} k_{ij}(\ket{i}\bra{j} + \ket{j}\bra{i}) + \sum_i \epsilon_i \ket{i}\bra{i}
\end{equation}

donde $k_{ij}$ representa los acoplamientos karmicos entre estados (sujeto a Paradoja~\ref{paradox:karma}) y $\epsilon_i$ los potenciales intrinsecos de cada estado.

\subsection{Los Seis Bardos como Transiciones Cuanticas}

Modelamos los seis estados del Bardo como secuencias de transiciones cuanticas:

\begin{enumerate}
    \item \textbf{Bardo del Momento de la Muerte (Chikhai Bardo)}: $\ket{2} \otimes \ket{k}$
    \item \textbf{Bardo de la Realidad (Chonyid Bardo)}: $\sum_k c_k\ket{k}$
    \item \textbf{Bardo del Devenir (Sidpa Bardo)}: $\ket{0} \leftarrow$ Medida
\end{enumerate}

\subsection{Genesis Conceptual: Del ERROR 505 al Qutrit Cuantico}

El punto de inflexion conceptual surgio del analisis de las clasificaciones digitales antropomorficas aplicadas a estados de conciencia post-mortem. La identificacion de "ERROR 505" como "Falta de reconocimiento de deidad" revelaba una limitacion fundamental en los modelos computacionales clasicos.

\subsubsection{Limitacion del Paradigma Binario}

La interpretacion como "error" emergia de un marco binario incapaz de representar:
\begin{itemize}
    \item Estados de superposicion cuantica no colapsados
    \item La vacuidad (śūnyatā) como estado fundamental
    \item Potencialidad kármica no actualizada
\end{itemize}

\subsubsection{Transición al Modelo Cuantico}

La resolución requirió trascender la logica booleana mediante:
\begin{equation}
\mathcal{H}_{\text{Bardo}} = \alpha\ket{0} + \beta\ket{1} + \gamma\ket{2}
\end{equation}
donde $\ket{2}$ señala hacia la vacuidad fundamental, no un estado de error.

Esta transicion paradigmatica permitio reinterpretar los "errores" como ventanas a estados de maxima potencialidad cuantica donde el karma puede reprogramarse.

% Metodología
\section{Metodologia: Implementacion Computacional}

\subsection{Arquitectura del Sistema de Simulacion}

Implementamos el sistema utilizando Python 3.11 con las siguientes bibliotecas principales:

\begin{lstlisting}[caption={Configuración del entorno de simulación cuántica}, label={lst:quantum_setup}]
import numpy as np
import qutip as qt
from scipy.linalg import expm
import matplotlib.pyplot as plt
from mpl_toolkits.mplot3d import Axes3D
import seaborn as sns

class BardoQuantumSystem:
    """Sistema cuantico con reflexividad epistemologica"""
    
    def __init__(self, karma_params=None):
        self.karma_params = karma_params or {
            'clarity': 0.8, 'attachment': 0.3, 
            'compassion': 0.9, 'wisdom': 0.7
        }
        self.dim = 3
        self.operators = self._create_operators()
        self.current_state = qt.basis(self.dim, 2)
        
        # Documentar limitaciones del modelo
        self.model_limitations = {
            'karma_reification': 
                'Parametros numericos reifican flujo impermanente',
            'temporal_assumption': 
                'Tiempo t es convencion, no realidad ultima',
            'measurement_duality': 
                'Mantiene marco observador-observado'
        }
    
    def _create_operators(self):
        """Crea operadores cuanticos fundamentales"""
        # Estados base
        kets = [qt.basis(3, i) for i in range(3)]
        
        # Proyectores P0, P1, P2
        P = {f'P{i}': kets[i] * kets[i].dag() for i in range(3)}
        
        # Operadores de transicion
        S01 = kets[0] * kets[1].dag()
        S12 = kets[1] * kets[2].dag()
        S20 = kets[2] * kets[0].dag()
        
        # Hamiltoniano base
        H0 = 0.1*P['P0'] + 0.2*P['P1'] + 0.3*P['P2']
        
        # Operador karmico (sujeto a Paradoja 1)
        K = (self.karma_params['attachment'] * (S01 + S01.dag()) +
             self.karma_params['clarity'] * (S12 + S12.dag()) +
             self.karma_params['compassion'] * (S20 + S20.dag()))
        
        P.update({'S01': S01, 'S12': S12, 'S20': S20, 'H0': H0, 'K': K})
        return P
\end{lstlisting}

\subsection{Algoritmo de Evolucion Temporal}

El algoritmo principal simula la evolucion completa a traves de los estados del Bardo:

\begin{lstlisting}[caption={Algoritmo de evolución del Bardo}, label={lst:bardo_evolution}]
def simulate_bardo_transition(self, time_steps=1000,
                            attention_function='logistic'):
    """Simula transicion con documentacion de asunciones"""
    
    times = np.linspace(0, 4*np.pi, time_steps)
    results = {
        'probabilities': [],
        'coherence': [],
        'purity': [],
        'states': [],
        'epistemic_notes': []
    }
    
    current_state = self.current_state
    
    for t in times:
        # Factor de atencion (convencion temporal)
        attention = self._attention_evolution(t, attention_function)
        
        # Hamiltoniano efectivo
        H_eff = self.operators['H0'] + attention * self.operators['K']
        
        # Evolucion unitaria incremental
        dt = times[1] - times[0] if len(times) > 1 else 0.01
        U = (-1j * dt * H_eff).expm()
        evolved_state = U * current_state
        current_state = evolved_state
        
        # Calcular probabilidades usando proyectores
        probs = [qt.expect(self.operators[f'P{i}'], evolved_state)
                for i in range(self.dim)]
        
        coherence = self._calculate_coherence(evolved_state)
        purity = self._calculate_purity(evolved_state)
        
        results['probabilities'].append(probs)
        results['coherence'].append(coherence)
        results['purity'].append(purity)
        results['states'].append(evolved_state)
        
        # Nota epistemologica cada 100 pasos
        if len(results['states']) % 100 == 0:
            note = self._generate_epistemic_note(evolved_state, t)
            results['epistemic_notes'].append(note)
        
        current_state = evolved_state
    
    return results, times

def _generate_epistemic_note(self, state, time):
    """Genera nota sobre limites del modelo en este punto"""
    probs = [qt.expect(self.operators[f'P{i}'], state) 
             for i in range(3)]
    dominant = np.argmax(probs)
    
    notes = {
        0: f"t={time:.2f}: Alta P(|0>) señala manifestacion, "
           f"pero forma es vacia",
        1: f"t={time:.2f}: P(|1>) alto indica potencial, "
           f"no karma sustancial",
        2: f"t={time:.2f}: P(|2>) alto apunta a sunyata, "
           f"no la describe"
    }
    return notes[dominant]
\end{lstlisting}

% Resultados
\section{Resultados y Simulaciones}

\subsection{Evolucion Temporal de Probabilidades}

\begin{figure}[h]
    \centering
    \includegraphics[width=0.9\textwidth]{figures/state_evolution.pdf}
    \caption{
        Evolucion temporal de probabilidades y metricas cuanticas en el sistema Bardo (Nivel Convencional).
        (A) Probabilidades de los estados fundamentales: Samsara ($\ket{0}$), Potencial Karmico ($\ket{1}$) y señalador de Vacuidad ($\ket{2}$).
        (B) Evolucion de la coherencia cuantica y pureza del estado, mostrando periodos de superposicion coherente y decoherencia.
        \textbf{Nota epistemológica:} Estas trayectorias son formalmente válidas pero no describen experiencia contemplativa directa.
    }
    \label{fig:state_evolution}
\end{figure}

\subsection{Analisis de Coherencia Cuantica}

La coherencia cuantica se mantiene durante las transiciones entre Bardos, con patrones caracteristicos:

\begin{equation}
C(\rho) = \sum_{i\neq j} |\rho_{ij}|
\end{equation}

\begin{epistemic_warning}{Interpretación de Coherencia}
La coherencia cuántica matemática es ANÁLOGA (no idéntica) a la "interpenetración no-dual" fenomenológica. El número $C(\rho)$ no mide directamente la claridad contemplativa (\emph{prajñā}), sino que señala hacia ella como correlato formal.
\end{epistemic_warning}

\begin{table}[h]
\centering
\caption{Metricas de coherencia por estado del Bardo (Nivel Convencional)}
\begin{tabular}{lccc}
\toprule
\textbf{Estado del Bardo} & \textbf{Coherencia} & \textbf{Pureza} & \textbf{Entropia} \\
\midrule
Chikhai Bardo & 0.95 $\pm$ 0.02 & 0.98 $\pm$ 0.01 & 0.12 $\pm$ 0.03 \\
Chonyid Bardo & 0.87 $\pm$ 0.04 & 0.92 $\pm$ 0.03 & 0.28 $\pm$ 0.05 \\
Sidpa Bardo & 0.45 $\pm$ 0.07 & 0.78 $\pm$ 0.06 & 0.65 $\pm$ 0.08 \\
\bottomrule
\end{tabular}
\end{table}

\subsection{Visualizacion de Transiciones Karmicas}

\begin{lstlisting}[caption={Generación de visualizaciones científicas}, label={lst:visualization}]
def create_comprehensive_visualization(results, times):
    """Crea visualizaciones con notas epistemologicas"""
    
    fig = plt.figure(figsize=(20, 12))
    
    # 1. Evolucion de probabilidades
    ax1 = fig.add_subplot(2, 3, 1)
    probabilities = np.array(results['probabilities'])
    ax1.plot(times, probabilities[:, 0], 
             label='$|0\\rangle$ Samsara', linewidth=2)
    ax1.plot(times, probabilities[:, 1], 
             label='$|1\\rangle$ Karmico', linewidth=2)
    ax1.plot(times, probabilities[:, 2], 
             label='$|2\\rangle$ Vacuidad', linewidth=2)
    ax1.set_xlabel('Tiempo')
    ax1.set_ylabel('Probabilidad')
    ax1.legend()
    ax1.grid(True, alpha=0.3)
    
    # 2. Coherencia cuantica
    ax2 = fig.add_subplot(2, 3, 2)
    ax2.plot(times, results['coherence'], 
             color='purple', linewidth=2)
    ax2.set_xlabel('Tiempo')
    ax2.set_ylabel('Coherencia Cuantica')
    ax2.grid(True, alpha=0.3)
    
    # 3. Esfera de Bloch 3D
    ax3 = fig.add_subplot(2, 3, 3, projection='3d')
    self._plot_bloch_sphere(results['states'], ax3)
    
    # Agregar nota epistemologica
    fig.text(0.5, 0.02, 
             'Nivel Convencional: Metricas formalmente validas',
             ha='center', fontsize=10, style='italic', color='red')
    
    plt.tight_layout()
    return fig
\end{lstlisting}

% Discusión
\section{Discusion: Implicaciones Interdisciplinarias}

\subsection{Validacion de la Hipotesis Central}

Nuestros resultados demuestran que (en el nivel convencional):

\begin{enumerate}
    \item El modelo de qutrits puede representar efectivamente la no-dualidad de la vacuidad \textit{como estructura lógica}
    \item Las transiciones entre estados del Bardo siguen dinamicas cuanticas coherentes \textit{en tanto analogía formal}
    \item El "ERROR 505" metaforico corresponde matematicamente a estados de superposicion no colapsada \textit{señalando hacia potencialidad}
\end{enumerate}

\begin{epistemic_warning}{Alcance de la Validación}
La "validación" es interna al modelo matemático. No afirmamos que el Bardo Thodol "sea" un algoritmo cuántico, sino que el formalismo cuántico puede usarse como \emph{upāya} para explorar su estructura lógica. La experiencia contemplativa post-mortem permanece fuera del alcance del modelo.
\end{epistemic_warning}

\subsection{Comparacion con Modelos Clasicos}

\begin{table}[h]
\centering
\caption{Comparacion entre modelos clasicos y cuanticos}
\begin{tabular}{lcc}
\toprule
\textbf{Caracteristica} & \textbf{Modelo Clasico} & \textbf{Modelo Cuantico} \\
\midrule
Representacion de vacuidad & ERROR 505 & Estado $\ket{2}$ señalador \\
Estados superpuestos & No posible & Fundamental \\
Naturaleza probabilistica & Simulada & Intrinseca \\
Transiciones no-locales & No & Si (analogicamente) \\
Coherencia temporal & No & Si \\
\textbf{Paradojas documentadas} & \textbf{Ignoradas} & \textbf{Explicitadas} \\
\bottomrule
\end{tabular}
\end{table}

\subsection{Implicaciones para la Ciencia de la Conciencia}

Nuestro trabajo sugiere (como hipótesis exploratoria, no afirmación ontológica):

\begin{itemize}
    \item Los estados de conciencia podrian seguir dinamicas cuanticas \textit{en ciertos aspectos estructurales}
    \item La meditacion profunda podria afectar parametros de coherencia cuantica \textit{medibles neurofisiológicamente}
    \item Los modelos computacionales de conciencia deben considerar frameworks cuanticos \textit{junto con explicitación de límites}
\end{itemize}

\subsection{Limitaciones Reconocidas del Proyecto}

\begin{enumerate}
    \item \textbf{Brecha fenomenológica}: El modelo no captura la experiencia directa (\emph{pratyakṣa}) de estados bardos
    \item \textbf{Reduccionismo paramétrico}: Karma cuantificado contradice su naturaleza de proceso interdependiente
    \item \textbf{Temporalidad artificial}: Tiempo matemático no refleja atemporalidad de \emph{samādhi}
    \item \textbf{Dualismo observacional}: Mantiene separación medidor-medido ausente en \emph{rigpa}
    \item \textbf{Cosificación de vacuidad}: $\ket{2}$ como vector contradice \emph{niḥsvabhāva}
\end{enumerate}

Estas limitaciones no son "problemas a resolver" sino características inherentes al modelado matemático de fenómenos contemplativos.

% Conclusión
\section{Conclusion y Trabajo Futuro}

\subsection{Conclusiones Principales}

\begin{enumerate}
    \item Hemos demostrado la viabilidad de modelar \textit{estructuralmente} estados de conciencia del Bardo Thodol usando sistemas cuanticos
    \item El enfoque de qutrits supera limitaciones de modelos binarios \textit{en el nivel de lógica formal}
    \item La vacuidad (sunyata) encuentra representacion matematica natural en superposiciones cuanticas \textit{como analogía, no identidad}
    \item Las dinamicas karmicas pueden ser implementadas como operadores cuanticos \textit{con reconocimiento explícito de reificación}
    \item \textbf{La explicitación de paradojas irresolibles} convierte el proyecto en metamodelo reflexivo
\end{enumerate}

\subsection{Marco Metodologico: Las Tres Verdades Aplicadas}

\begin{table}[h]
\centering
\caption{Aplicación del método de las Dos Verdades al modelado}
\begin{tabular}{p{3.5cm}p{5.5cm}p{5cm}}
\toprule
\textbf{Nivel} & \textbf{Qué afirma} & \textbf{Qué NO afirma} \\
\midrule
\textbf{Convencional} (\emph{saṃvṛti}) & Matemáticas cuánticas válidas formalmente & Que describen realidad última \\
\textbf{Último} (\emph{paramārtha}) & Fenómenos carecen de naturaleza inherente & Que matemáticas sean inútiles \\
\textbf{Pedagógico} (\emph{upāya}) & Modelo útil para explorar estructuras & Que sea descripción ontológica \\
\bottomrule
\end{tabular}
\end{table}

\subsection{Direcciones Futuras}

\begin{itemize}
    \item \textbf{Validacion Experimental}: Integracion con datos de meditacion avanzada y EEG, \textit{con advertencia de que correlatos neurales no son la experiencia}
    \item \textbf{Hardware Cuantico}: Implementacion en procesadores cuanticos reales (IBM Q, Rigetti), \textit{como demostración de viabilidad computacional}
    \item \textbf{Modelos Extendidos}: Generalizacion a sistemas de mas estados y dimensiones, \textit{manteniendo transparencia epistemológica}
    \item \textbf{Aplicaciones Clinicas}: Potenciales aplicaciones en terapia y estados alterados de conciencia, \textit{sin reduccionismo psicologicista}
    \item \textbf{Diálogo contemplativo-científico}: Validación con practicantes avanzados sobre utilidad heurística del modelo
\end{itemize}

\subsection{Impacto Cientifico y Filosofico}

Este trabajo establece un puente entre la sabiduria contemplativa ancestral y la ciencia computacional moderna, abriendo nuevas vias para la investigacion interdisciplinaria en:

\begin{itemize}
    \item Filosofia de la mente y ciencia cognitiva \textit{con epistemología no-reduccionista}
    \item Computacion cuantica y teoria de la informacion \textit{aplicada a fenomenología}
    \item Estudios contemplativos y neurofenomenologia \textit{con respeto a irreductibilidad experiencial}
    \item \textbf{Metamodelado reflexivo}: Modelos que incorporan crítica a sí mismos
\end{itemize}

\subsection{Reflexión Final: El Dedo y la Luna}

Como enseña el \emph{Laṅkāvatāra Sūtra}:

\begin{quote}
\textit{"Las palabras y las enseñanzas son como un dedo apuntando a la luna. El dedo puede indicar dónde está la luna, pero el dedo no es la luna. Para ver la luna, es necesario mirar más allá del dedo."}
\end{quote}

Este modelo computacional es el dedo. La experiencia directa de los estados del Bardo es la luna. No confundir uno con otro es la sabiduría que permite usar el modelo efectivamente.

% Apéndices
\begin{appendices}
\section{Implementacion Completa del Codigo}

\subsection{Clase Principal del Sistema}

\begin{lstlisting}[caption={Sistema BardoQuantumSystem completo}, label={lst:complete_system}]
class QuantumMetrics:
    """Clase para calcular metricas cuanticas avanzadas"""
    
    @staticmethod
    def coherence(state):
        """Calcula coherencia cuantica (norma l1 fuera de diagonal)"""
        if state.type == 'ket':
            rho = state * state.dag()
        else:
            rho = state
        rho_array = rho.full()
        n = rho_array.shape[0]
        coh = 0.0
        for i in range(n):
            for j in range(n):
                if i != j:
                    coh += abs(rho_array[i, j])
        return coh
    
    @staticmethod
    def purity(state):
        """Calcula pureza del estado: Tr(rho^2)"""
        if state.type == 'ket':
            return 1.0
        else:
            rho = state
            return (rho * rho).tr().real
    
    @staticmethod
    def von_neumann_entropy(state):
        """Calcula entropia de Von Neumann: -Tr(rho log2 rho)"""
        if state.type == 'ket':
            rho = state * state.dag()
        else:
            rho = state
        eigvals = rho.eigenvalues()
        entropy = 0.0
        for v in eigvals:
            if v > 0:
                entropy -= v * np.log2(v)
        return entropy


class QuantumAnalytics:
    """Sistema centralizado de analisis cuantico"""
    
    @staticmethod
    def analyze_transitions(probabilities, threshold=0.1):
        """Analisis unificado de transiciones entre estados"""
        probs = np.array(probabilities)
        transitions = []
        
        for i in range(1, len(probs)):
            changes = np.abs(probs[i] - probs[i-1])
            max_change = np.max(changes)
            
            if max_change > threshold:
                transitions.append({
                    'time_index': i,
                    'magnitude': float(max_change),
                    'from_state': int(np.argmax(probs[i-1])),
                    'to_state': int(np.argmax(probs[i])),
                    'change_vector': changes.tolist()
                })
        
        return transitions
    
    @staticmethod
    def find_dominant_state(probabilities):
        """Analisis unificado de estado dominante"""
        probs = np.array(probabilities)
        dominant_states = np.argmax(probs, axis=1)
        total_steps = len(dominant_states)
        
        return {
            'dominant_states': dominant_states.tolist(),
            'time_in_samsara': int(np.sum(dominant_states == 0)),
            'time_in_karmic': int(np.sum(dominant_states == 1)),
            'time_in_void': int(np.sum(dominant_states == 2)),
            'dominance_ratio': {
                'samsara': float(np.sum(dominant_states == 0) / total_steps),
                'karmic': float(np.sum(dominant_states == 1) / total_steps),
                'void': float(np.sum(dominant_states == 2) / total_steps)
            }
        }


class BardoQuantumSystem:
    """
    Sistema completo de simulacion cuantica del Bardo Thodol
    CON DOCUMENTACION EXPLICITA DE LIMITACIONES
    """
    
    def __init__(self, **parameters):
        self.set_parameters(parameters)
        self.initialize_quantum_system()
        self.metrics = QuantumMetrics()
        self.analytics = QuantumAnalytics()
        
        # Documentar paradojas del modelo
        self.epistemic_warnings = {
            'karma_quantification': 
                'Parametros numericos reifican karma (Paradoja 1)',
            'sunyata_vector': 
                'Vector |2> cosifica vacuidad (Paradoja 2)',
            'temporal_parameter': 
                'Tiempo t es convencion matematica (Paradoja 3)',
            'measurement_duality': 
                'Mantiene marco sujeto-objeto (Paradoja 4)'
        }
    
    def set_parameters(self, params):
        """Configura parametros del sistema"""
        self.karma_params = params.get('karma_params', {
            'clarity': 0.8,
            'attachment': 0.3,
            'compassion': 0.9,
            'wisdom': 0.7
        })
        self.time_parameters = params.get('time_params', {
            'total_time': 4*np.pi,
            'steps': 1000
        })
    
    def initialize_quantum_system(self):
        """Inicializa el sistema cuantico base"""
        self.dimension = 3
        self.states = {
            'samsara': qt.basis(3, 0),
            'karmic': qt.basis(3, 1),
            'void': qt.basis(3, 2)
        }
        self.operators = self._create_operators()
        self.current_state = self.states['void']
    
    def _create_operators(self):
        """Crea los operadores cuanticos para el sistema"""
        # Operadores de proyeccion
        P0 = qt.basis(3, 0) * qt.basis(3, 0).dag()
        P1 = qt.basis(3, 1) * qt.basis(3, 1).dag()
        P2 = qt.basis(3, 2) * qt.basis(3, 2).dag()
        
        # Operadores de transicion
        S01 = qt.basis(3, 0) * qt.basis(3, 1).dag()
        S12 = qt.basis(3, 1) * qt.basis(3, 2).dag()
        S20 = qt.basis(3, 2) * qt.basis(3, 0).dag()
        
        # Hamiltoniano base
        H0 = P0 * 0.1 + P1 * 0.2 + P2 * 0.3
        
        # Operador karmico
        K = self.karma_params['attachment'] * (S01 + S01.dag()) + \
            self.karma_params['clarity'] * (S12 + S12.dag()) + \
            self.karma_params['compassion'] * (S20 + S20.dag())
        
        return {
            'P0': P0, 'P1': P1, 'P2': P2,
            'S01': S01, 'S12': S12, 'S20': S20,
            'H0': H0, 'K': K
        }
    
    def simulate_bardo_transition(self, time_steps=1000, 
                                  attention_function='logistic'):
        """Simula la transicion completa"""
        times = np.linspace(0, self.time_parameters['total_time'], 
                          time_steps)
        results = {
            'probabilities': [],
            'coherence': [],
            'purity': [],
            'states': []
        }
        
        current_state = self.current_state
        
        for t in times:
            attention = self._attention_evolution(t, attention_function)
            H_eff = self.operators['H0'] + attention * self.operators['K']
            U = (-1j * t * H_eff).expm()
            evolved_state = U * current_state
            
            probs = [qt.expect(self.operators[f'P{i}'], evolved_state) 
                    for i in range(3)]
            coherence = self.metrics.coherence(evolved_state)
            purity = self.metrics.purity(evolved_state)
            
            results['probabilities'].append(probs)
            results['coherence'].append(coherence)
            results['purity'].append(purity)
            results['states'].append(evolved_state)
            
            current_state = evolved_state
        
        return results, times
    
    def _attention_evolution(self, t, attention_function='logistic'):
        """Evolucion de la atencion en el tiempo"""
        if attention_function == 'logistic':
            return 1.0 / (1.0 + np.exp(-0.5 * (t - 2*np.pi)))
        elif attention_function == 'sinusoidal':
            return 0.5 * (1.0 + np.sin(t))
        else:
            return 1.0
    
    def run_complete_simulation(self):
        """Ejecuta simulacion completa con analisis"""
        results, times = self.simulate_bardo_transition()
        probs_array = np.array(results['probabilities'])
        
        analysis_report = {
            'final_state_classification': self._classify_final_state(
                results['states'][-1]
            ),
            'transitions': self.analytics.analyze_transitions(probs_array),
            'dominant_state_analysis': 
                self.analytics.find_dominant_state(probs_array),
            'quantum_metrics': {
                'avg_coherence': float(np.mean(results['coherence'])),
                'avg_purity': float(np.mean(results['purity'])),
                'final_entropy': self.metrics.von_neumann_entropy(
                    results['states'][-1]
                )
            },
            'epistemic_warnings': self.epistemic_warnings
        }
        
        return results, times, analysis_report
    
    def _classify_final_state(self, state):
        """Clasifica el estado final segun las probabilidades"""
        probs = [float(qt.expect(self.operators[f'P{i}'], state)) 
                for i in range(3)]
        max_prob_index = np.argmax(probs)
        states_names = ['Samsara', 'Karmico', 'Vacuidad']
        
        return {
            'dominant_state': states_names[max_prob_index],
            'probabilities': probs,
            'certainty': float(max(probs)),
            'note': 'Clasificacion en nivel convencional (samvrti-satya)'
        }
\end{lstlisting}

\end{appendices}

\begin{thebibliography}{99}
\bibitem{bardo1} Fremantle, F. (2001). \emph{The Tibetan Book of the Dead}. Shambhala Publications.
\bibitem{quantum1} Nielsen, M. A., \& Chuang, I. L. (2010). \emph{Quantum Computation and Quantum Information}. Cambridge University Press.
\bibitem{consciousness1} Hameroff, S., \& Penrose, R. (2014). Consciousness in the universe: A review of the 'Orch OR' theory. \emph{Physics of Life Reviews}, 11(1), 39-78.
\bibitem{buddhism1} Wallace, B. A. (2007). \emph{Contemplative Science: Where Buddhism and Neuroscience Converge}. Columbia University Press.
\bibitem{madhyamaka1} Nāgārjuna. (2013). \emph{The Fundamental Wisdom of the Middle Way: Nāgārjuna's Mūlamadhyamakakārikā}. Oxford University Press.
\bibitem{qutrit1} Lanyon, B. P., et al. (2008). Manipulating biphotonic qutrits. \emph{Physical Review Letters}, 100(6), 060504.
\bibitem{computation1} Tegmark, M. (2000). Importance of quantum decoherence in brain processes. \emph{Physical Review E}, 61(4), 4194.
\bibitem{epistemology1} Varela, F. J., Thompson, E., \& Rosch, E. (2016). \emph{The Embodied Mind: Cognitive Science and Human Experience}. MIT Press.
\end{thebibliography}

\end{document}