\documentclass[12pt,a4paper]{article}

% ==== Paquetes ====
\usepackage[utf8]{inputenc}
\usepackage{amsmath, amssymb}
\usepackage{graphicx}
\usepackage{listings}
\usepackage{xcolor}
\usepackage{hyperref}
\usepackage{csquotes}
\usepackage{geometry}
\usepackage{biblatex}
\addbibresource{references.bib}

\geometry{margin=2.5cm}

% ==== Configuración para código ====
\lstset{
  basicstyle=\ttfamily\footnotesize,
  breaklines=true,
  frame=single,
  backgroundcolor=\color{gray!10}
}

% ==== Front Matter ====
\title{\textbf{Simulación Cuántica de Estados Post-Mortem del Bardo Thödol}}
\author{Horacio Hamann \\ \texttt{https://github.com/arathorian}}
\date{Septiembre 2025}

\begin{document}

\maketitle

\begin{abstract}
Este trabajo explora el \emph{Bardo Thödol} (Libro Tibetano de los Muertos) 
desde un marco interdisciplinario que combina física cuántica, neurociencia 
y lingüística. Se propone un modelo computacional para simular los estados 
de conciencia descritos en el texto, integrando métricas neurofisiológicas, 
circuitos cuánticos y patrones lingüísticos. El objetivo es establecer un 
puente entre la tradición contemplativa y la ciencia moderna, proponiendo 
tanto implementaciones en hardware cuántico como vías de validación experimental.
\end{abstract}

\tableofcontents

% === 1. INTRODUCCIÓN ===
\section{Introducción}
Contexto histórico del \emph{Bardo Thödol}. Motivación científica: de texto 
ritual a modelo cuántico. Estado del arte en teorías de conciencia cuántica. 
Objetivos del proyecto.

% === 2. MARCO TEÓRICO ===
\section{Marco Teórico}

\subsection{Estados de Conciencia en el Bardo}
Entrenamiento en vida (Tukdam y simulaciones de conciencia). Estados post-mortem 
como fases transitorias. Evidencia neurocientífica \cite{wisconsin2022gamma, mindlife2023tukdam}.

\subsection{Modelado Matemático}
\begin{align}
| \psi_{bardo} \rangle = \frac{1}{\sqrt{2}} (|vida\rangle + |muerte\rangle)
\end{align}
Qutrits $(|0\rangle, |1\rangle, |2\rangle)$ para incluir vacuidad. 
Parámetros kármicos dinámicos $\kappa(t), \alpha(t)$. 
Matrices CPT-duales.

\subsection{Lingüística y Algoritmos Binarios}
Patrones Fibonacci y 3-6-9 (Tesla). Mandalas como mapas geométricos cuánticos 
\cite{kyoto2024linguistics, watanabe2024mandalas}.

% === 3. IMPLEMENTACIÓN COMPUTACIONAL ===
\section{Implementación Computacional}

\subsection{Simulación Clásica (Python)}
Ejemplo de activación de deidad pacífica:
\begin{lstlisting}[language=Python]
deidad_pacifica = {
  "frecuencia": "528Hz",
  "receptores": ["NMDA_GluN2B", "5-HT2A"],
  "efecto": "reprogramacion karmica"
}
\end{lstlisting}

\subsection{Circuitos Cuánticos (Qiskit)}
Diseño de circuito cuántico en IBM Quantum \cite{ibm2025qiskit}. 
Métricas: discordia cuántica $D_q > 0.45$, negatividad. 
Emulación de qutrits con pares de qubits.

\subsection{Resultados Gráficos}
\begin{itemize}
  \item \texttt{brain\_coherence\_study.png} → coherencia gamma.
  \item \texttt{quantum\_circuit\_bardo\_simulation.png} → circuito IBM.
  \item \texttt{biophotons\_coherence.png} → biofotones coherentes.
\end{itemize}

% === 4. VALIDACIÓN EXPERIMENTAL ===
\section{Vías de Validación Experimental}
Neuroimagen (EEG, fMRI) en meditadores avanzados \cite{wisconsin2022gamma}.  
Estudios de Tukdam post-mortem \cite{mindlife2023tukdam}.  
Medición de biofotones coherentes \cite{stanford2023clearlight}.  
Implementación piloto en hardware cuántico (IBM, IonQ, Rigetti).  
Colaboraciones interdisciplinarias.

% === 5. COMPARATIVA CON TEORÍAS ===
\section{Comparativa con Teorías de Conciencia}
Modelo Orch-OR de Penrose y Hameroff \cite{hameroff2014consciousness}.  
Similitudes y diferencias con el presente enfoque.  
Aportes originales: qutrits y parámetros kármicos dinámicos.

% === 6. DESAFÍOS EN HARDWARE ===
\section{Desafíos de Implementación en Hardware Real}
Qutrits vs qubits. Corrección de errores cuánticos.  
Comparativa de plataformas (IBM, IonQ, Rigetti).  

% === 7. ÉTICA Y CULTURA ===
\section{Consideraciones Éticas y Culturales}
Respeto a la tradición tibetana. Colaboración intercultural.  
Limitaciones del modelado computacional.  

% === 8. APLICACIONES ===
\section{Aplicaciones e Implicaciones}

\subsection{Científicas}
Conciencia como hardware cuántico \cite{hameroff2014consciousness}.

\subsection{Tecnológicas}
API de conciencia y afterlife VR \cite{mit2024symmetry}.

\subsection{Médicas y Sociales}
Terapia de fin de vida, biomarcadores de conciencia \cite{stanford2023clearlight}.

\subsection{Filosóficas}
Reinterpretación del Bardo como manual cuántico \cite{penrose1994shadows}.

% === 9. CONCLUSIÓN ===
\section{Conclusión Críptica}
“El \emph{Bardo Thödol} no es un texto religioso, 
sino un manual de usuario de la conciencia para su transición a estados cuánticos.”

% === APÉNDICE ===
\appendix
\section{Procesamiento Lingüístico Cuántico del Bardo Thödol}
Patrones binarios y Fibonacci en versos \cite{kyoto2024linguistics}.  
Modelado en espacios de Hilbert.  
Potenciales aplicaciones del QNLP en textos antiguos.

% === REFERENCIAS ===
\printbibliography

\end{document}
